\chapter{Templates}

\begin{description}
\item[\textcolor{green}{Author}] Author of the approaches description  \todo{Name -  Company}
\item[\textcolor{blue}{Assessor 1}] First assessor of the approaches \todo{Name - Company}
\item[\textcolor{magenta}{Assessor 2}] Second assessor of the approaches \todo{Name - Company}
\end{description}

In the sequel, main text is under the responsibilities of the author.

\begin{author_comment}
Author can add comments using this format
\end{author_comment}

\begin{assessor1}
First assessor can add comments using this format
\end{assessor1}

\begin{assessor2}
Second assessor can add comments using this format
\end{assessor2}

When a note is required, please follow this list :
\begin{description}
\item[0] not recommended, not adapted, rejected
\item[1] weakly recommended, adapted after major improvements, weakly rejected
\item[2] recommended, adapted (with light improvements if necessary)  weakly accepted
\item[3] highly recommended, well adapted,strongly accepted
\end{description}

\section{Presentation}
This section gives a quick presentation of the tool.

\begin{description}
\item[Name] Name of the approach
\item[Web site] If available, how to find information
\item[Licence] Kind of licence
\end{description}

\paragraph{Abstract} Short abstract on the tool (5 lines max)

\paragraph{Publications} short list of publication on the tool (5 max)


\section{Main usage of the tool}
\label{main_usage}

This section discusses the main usage of the tool.

Which task are covered by the tool ?


\begin{tabular}{|l | c | c | c | c|}
\hline
& \textcolor{green}{Author} & \textcolor{blue}{Assessor 1} & \textcolor{magenta}{Assessor 2} & Total \\
\hline 
Modelling support & & & &  \\
\hline
Automatic translation  & & & & \\
\hline
Code Generation  & & & & \\
\hline
Model verification & & & & \\
\hline
Test generation & & & & \\
\hline
Simulation, execution, debugging & & & & \\
\hline
Formal proof & & & & \\
\hline
\end{tabular}

\paragraph{Modelling support}
Does the tool provide a  textual or a graphical editor ?

\paragraph{Automatic translation and code generation}
Which translation or code generation is supported by the tool ?

\paragraph{Model verification}
Which verification on models are provided by the tool?

\paragraph{Test generation}
Does the tool allow to generate tests ? For  which purpose ?

\paragraph{Simulation, execution, debugging}
Does the tool allow to simulate or to debbug step by step a model or a code ?

\paragraph{Formal proof}
Does the tool allow formal proof ?  How ?



\section{Use of the tool}


According WP2 requirements, give a note for characteristics of the use of the tool (from 0 to 3) :

\begin{tabular}{|l | c | c | c | c|}
\hline
& \textcolor{green}{Author} & \textcolor{blue}{Assessor 1} & \textcolor{magenta}{Assessor 2} & Total \\
\hline 
Open Source (D2.6-01-029) & & & &  \\
\hline 
Portability to operating systems (D2.6-01-030) & & & &  \\
\hline
Cooperation of tools (D2.6-01-031) & & & &  \\
\hline
Robustness (D2.6-01-034) & & & & \\
\hline
Modularity (D2.6-01-034.01) & & & & \\
\hline
Documentation management (D.2.6-01-034.02) & & & & \\
\hline
Distributed software development (D.2.6-01-034.03)  & & & & \\
\hline
Issue tracking (D.2.6-01-034.04) & & & & \\
\hline
Differences between models (D.2.6-01-034.05) & & & & \\
\hline
Version management (D.2.6-01-034.06) & & & & \\
\hline
Concurrent version management (D.2.6-01-034.07) & & & & \\
\hline
Model-based version control (D.2.6-01-034.08) & & & & \\
\hline
Role traceability (D.2.6-01-034.09) & & & & \\
\hline
Safety version traceability (D.2.6-01-034.10) & & & & \\
\hline
Model traceability (D.2.6-01-035) & & & & \\
\hline
Tool chain integration & & & & \\
\hline
\end{tabular}

\section{Certifiability}

This section discusses how the tool can be classified according EN50128 requirements (D.2.6-X-49).


\begin{tabular}{|l | c | c | c | c|}
\hline
& \textcolor{green}{Author} & \textcolor{blue}{Assessor 1} & \textcolor{magenta}{Assessor 2} & Total \\
\hline 
Tool manual (D.2.6-01-42.02) & & & &  \\
\hline
Proof of correctness (D.2.6-01-42.03)   & & & & \\
\hline
Existing industrial  usage  & & & & \\
\hline
Model verification & & & & \\
\hline
Test generation & & & & \\
\hline
Simulation, execution, debugging & & & & \\
\hline
Formal proof & & & & \\
\hline
\end{tabular}

\paragraph{Other elements for tool certification}

\section{Other comments}
Please to  give free comments on the approach.

