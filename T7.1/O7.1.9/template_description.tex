\chapter{Template Description}
\label{sec:template}

\begin{description}
\item[\textcolor{green}{Author}] Author of the approaches description  \todo{Name -  Company}
\item[\textcolor{blue}{Assessor 1}] First assessor of the approaches \todo{Name - Company}
\item[\textcolor{magenta}{Assessor 2}] Second assessor of the approaches \todo{Name - Company}
\end{description}

In the sequel, main text is under the responsibilities of the author.

\begin{author_comment}
Author can add comments using this format at any place.
\end{author_comment}

\begin{assessor1}
First assessor can add comments using this format at any place.
\end{assessor1}

\begin{assessor2}
Second assessor can add comments using this format at any place.
\end{assessor2}

When a note is required, please follow this list :
\begin{description}
\item[0] not recommended, not adapted, rejected
\item[1] weakly recommended, adapted after major improvements, weakly rejected
\item[2] recommended, adapted (with light improvements if necessary)  weakly accepted
\item[3] highly recommended, well adapted,strongly accepted
\item[*] difficult to evaluate with a note (please add a comment under the table)
\end{description}

All the notes can be commented under each table.

\section{Presentation}

This section gives a quick presentation of the tool.

\begin{description}
\item[Name] \todo{Name of the approach and the tool}
\item[Web site] \todo{if available, how to  find information}
\item[License] \todo{Kind of license}
\end{description}

\paragraph{Abstract} Short abstract on the approach and tool (10 lines max)

\paragraph{Publications} Short list of publications on the approach (5 max)

\section{Use and usage of  of the Tool}
According WP2 requirements, give a note for characteristics of the use of the tool (from 0 to 3) :

\begin{tabular}{|l | c | c | c | c|}
\hline
& \textcolor{green}{Author} & \textcolor{blue}{Assessor 1} & \textcolor{magenta}{Assessor 2} & Total \\
\hline 
Open Source (D2.6-02-074) & & & &  \\
\hline 
Cooperation of tools (D2.6-02-076) & & & &  \\
\hline
Robustness (D2.6-02-078) & & & & \\
\hline
Modularity (D2.6-02-078.01) & & & & \\
\hline
Documentation management (D.2.6-02-078.02) & & & & \\
\hline
Distributed software development (D.2.6-02-078.03)  & & & & \\
\hline
Simultaneous Multi-users (D.2.6-02-078.04)   & & & & \\
\hline
Issue tracking (D.2.6-02-078.05) & & & & \\
\hline
Concurrent version development (D.2.6-02-078.08) & & & & \\
\hline
Model-based version control (D.2.6-02-078.09) & & & & \\
\hline
Role traceability (D.2.6-02-078.10) & & & & \\
\hline
Safety version traceability (D.2.6-02-078.11) & & & & \\
\hline
Scalability & & & & \\
\hline
\end{tabular}

The next table resume the capabilities of the tool platform.
\begin{enumerate}
\item Easy integration of a new tool in the tool platform.
\item How expert should we be ?
\item Mechanism for maintaining the tools within the tool platform.
\item Mechanism for bug tracking the tools within the tool platform.
\item Mechanism for updating the tools within the tool platform.
\item Mechanism to know the set of tool used.
\end{enumerate}

\begin{tabular}{|l | c | c | c | c|}
\hline
& \textcolor{green}{Author} & \textcolor{blue}{Assessor 1} & \textcolor{magenta}{Assessor 2} & Total \\
\hline Easy integration of a new tool &
  &                 &                  &\\
\hline Expertise &
  &                 &                  &\\
\hline Maintenance &
  &                 &                  &\\
\hline Bug tracking &
  &                 &                  &\\
\hline Update &
  &                 &                  &\\
\hline  Catalog &
  &                 &                  &\\
\hline
\end{tabular}


\section{Platform integration}
Capabilities of the tool platform to  be integrated to common operating systems.
\begin{enumerate}
\item  The tools chain shall be portable to common \gls{OS}.
\item   The tool chain shall run stable on all main \gls{OS}.
\item  The tool chain shall run with a good performance on all main \gls{OS}.
\item  Data produce on an \gls{OS} should be readable by an other \gls{OS}.
\end{enumerate}
\begin{tabular}{|l | c | c | c | c|}
  \hline
  & \textcolor{green}{Author} & \textcolor{blue}{Assessor 1} &  \textcolor{magenta}{Assessor 2} & Total \\
  \hline  Portable (R-WP2/D2.6-02-076) &
  &                 &                  &\\
  \hline   Stable (R-WP2/D2.6-02-076.01)&
  &                 &                  &\\
  \hline   Same performance (R-WP2/D2.6-02-076.02)&
  &                 &                  &\\
  \hline  Easy exchange (R-WP2/D2.6-02-076)&
  &                 &                  &\\
  \hline
\end{tabular}


\section{Data integration}
How the tool platform supports the data integration ?

\subsection{How to share data ?}
\begin{enumerate}
\item Files contains all info and are read/write when needed.
\item A central repository contains all data (maybe distributed in a
  set of files).
\item A common database is shared among tools
\item Tools communicate via messages.
\item A meta-model is defined to harmonized the data input/output
\item Representational State Transfer (REST), distributed architecture (like the web) and data access by an
  unique identifier (URL-like).
\end{enumerate}
\begin{tabular}{|l | c | c | c | c|} \hline
  & \textcolor{green}{Author} & \textcolor{blue}{Assessor 1} &  \textcolor{magenta}{Assessor 2} & Total \\
  \hline File based sharing &
  &                 &                  &\\
  \hline Shared repository &
  &                 &                  &\\
  \hline Database based sharing&
  &                 &                  &\\
  \hline Message Passing&
  &                 &                  &\\
  \hline Meta-Model based sharing &
  &                 &                  &\\
  \hline REST architecture style &
  &                 &                  &\\
  \hline
\end{tabular}


\subsection{Data definition}
\begin{enumerate}
\item Definition of a common data format
\item All possible input and output formats of a tool have to be
  documented.
\item Automatic parser/generator support compliant with a common format.
\item Open data formats shall be used for data exchange.
 
\end{enumerate}

\begin{tabular}{|l | c | c | c | c|} \hline
  & \textcolor{green}{Author} & \textcolor{blue}{Assessor 1} &  \textcolor{magenta}{Assessor 2} & Total \\
  \hline Common data format&
  &                 &                  &\\
  \hline Common data format documentation(R-WP2/D2.6-02-076.01) &
  &                 &                  &\\
  \hline Generator/parser &
  &                 &                  &\\
  \hline Open data format (R-WP2/D2.6-02-076.02)&
  &                 &                  &\\
  \hline
\end{tabular}

{\bf Does the tool platform define mechanisms to check the data produces
by a tool ?}

\section{Presentation integration}
The tool chain would be easier to use if the tool may share a common
"look and fell'.
\begin{enumerate}
\item The tool platform proposes a common \gls{GUI}.
\item The tool platform proposes some presentation guidelines
\item The tool platform provides a user interface creation toolkit
\item The tool platform integrates new tools as plug-in of the \gls{IDE}.
\end{enumerate}

\begin{tabular}{|l | c | c | c | c|} \hline
  & \textcolor{green}{Author} & \textcolor{blue}{Assessor 1} &  \textcolor{magenta}{Assessor 2} & Total \\
  \hline Common \gls{GUI}&
  &                 &                  &\\
  \hline Presentation guidelines &
  &                 &                  &\\
  \hline User interface toolkit &
  &                 &                  &\\
  \hline Plug-in support &
  &                 &                  &\\
  \hline
\end{tabular}


\section{Control integration}
The mechanism that allows tools to notify and activate others tools.
\begin{enumerate}
\item Tools may subscribe to some notification.
\item Tools may notify changes.
\item The change may be logged and traced.
\item The tool platform provides a version management.
\item Script based change control.
\end{enumerate}

\begin{tabular}{|l | c | c | c | c|} \hline
  & \textcolor{green}{Author} & \textcolor{blue}{Assessor 1} &  \textcolor{magenta}{Assessor 2} & Total \\
  \hline Service Subscriber&
  &                 &                  &\\
  \hline Service Notifier &
  &                 &                  &\\
  \hline Change traceability &
  &                 &                  &\\
  \hline Version Management &
  &                 &                  &\\
  \hline Script control &
  &                 &                  &\\
  \hline
\end{tabular}


\section{Process integration}
\subsection{Support for engineering process}
Support for a well-defined software engineering process. 
\begin{enumerate}
\item Process definition support.
\item Integrated process : the process is transparent to the users.
\item Process Management support.
\item Script based process: the process is implemented via scripts.
\item The defined tool chain  may be analyzed within the tool platform.
\end{enumerate}

\begin{tabular}{|l | c | c | c | c|} \hline
  & \textcolor{green}{Author} & \textcolor{blue}{Assessor 1} &  \textcolor{magenta}{Assessor 2} & Total \\
  \hline Process Definition &
  &                 &                  &\\
  \hline Process Management &
  &                 &                  &\\
  \hline Integrated Process &
  &                 &                  &\\
  \hline Script based process &
  &                 &                  &\\
  \hline Tool chain Analysis &
  &                 &                  &\\
  \hline
\end{tabular}



\subsection{Support for EN 50128 standard}
Specific mechanism related to EN 50128 standard
\begin{enumerate}
\item  Each tool in the tool chain shall be classified among T1, T2
  and T3 depending on its usage in the process.
\item  For T2 and T3 tools 7 , the choice of tools shall be justified,
  and the justification shall include how the tools failures are
  covered, avoided or taken into account (ref. to EN 50128 6.7.4.2).
\item  All T2 and T3 tools must be provided with their user manuals.
\item  For all T3 tool, the proof of correctness or the measure taken
  to guarantee the correctness of the output w.r.t. their
  specification and the inputs shall be provided. The tool platform
  provides output/input comparison  or other mechanism to guarantee
  the correctness.
\item Does the tool platform provides help to generate or to produce
  documentation about the tool chain such as manual, specification  EN
  50128 §6.7 ?
\item The tool platform helps to define the maintenance process of the
  tool chain.
\item The tool platform provides bug tracker for the tool chain.
\end{enumerate}

\begin{tabular}{|l | c | c | c | c|} \hline
  & \textcolor{green}{Author} & \textcolor{blue}{Assessor 1} &  \textcolor{magenta}{Assessor 2} & Total \\
  \hline Integrated Tools classification (R-WP2/D2.6-02-085)  &
  &                 &                  &\\
  \hline Document production (R-WP2/D2.6-01-042.01) &
  &                 &                  &\\
  \hline Automatic information on tools (R-WP2/D2.6-01-042.02) &
  &                 &                  &\\
  \hline Measure for correctness (R-WP2/D2.6-01-042.03) &
  &                 &                  &\\
  \hline Test of the tool platform &
  &                 &                  &\\
  \hline Tool chain documents  production &
  &                 &                  &\\
  \hline Tool chain maintenance process specification &
  &                 &                  &\\
 \hline Tool chain bug tracker &
  &                 &                  &\\
  \hline
\end{tabular}

\section{Tool chain Analysis}
Is the tool platform able to analyze the tool chain.
\begin{enumerate}
\item The tool platform provides a met-model to represents tool chain
\item The tool platform provides a graphical representation of the
  tool chain
\item The tool platform provides a textual description of the
  tool chain
\item The tool platform may help to analyze the tool chain by means
  of the tools inputs and outputs.
\item The tool platform allows the rearrangement and/or the extension
  of the tool chain
\item Previously define tool chain may be re-use and their confidence too.
\end{enumerate}

\begin{tabular}{|l | c | c | c | c|} \hline
  & \textcolor{green}{Author} & \textcolor{blue}{Assessor 1} &  \textcolor{magenta}{Assessor 2} & Total \\
  \hline Tool chain Meta-Model definition &
  &                 &                  &\\
  \hline Tool chain graphical representation &
  &                 &                  &\\
  \hline Tool chain textual representation &
  &                 &                  &\\
  \hline Tool's Input/output Analysis &
  &                 &                  &\\
  \hline Tool chain Analysis &
  &                 &                  &\\
  \hline Tool chain rearrangement and extension &
  &                 &                  &\\
  \hline Confidence preservation &
  &                 &                  &\\
  \hline
\end{tabular}


\section{Other comments}
Please to  give free comments on the approach.

%%% Local Variables: 
%%% mode: latex
%%% TeX-master: "Evaluation_Platform_against_WP2"
%%% End: 
