\chapter{Petri Nets}

\begin{description}
\item[\textcolor{green}{Author}] Author of the approaches description  Jan Welte (TU-BS)
\item[\textcolor{blue}{Assessor 1}] First assessor of the approaches Marc Antoni (SNCF)
\item[\textcolor{magenta}{Assessor 2}] Second assessor of the approaches Cyril Cornu (All4Tec)
\end{description}

In the sequel, main text is under the responsibilities of the author.

\begin{author_comment}
Author can add comments using this format at any place.
\end{author_comment}

\begin{assessor1}
First assessor can add comments using this format at any place.
\end{assessor1}

\begin{assessor2}
Second assessor can add comments using this format at any place.
\end{assessor2}


When a note is required, these have been given using the following list:
\begin{description}
\item[0] not recommended, not adapted, rejected
\item[1] weakly recommended, adapted after major improvements, weakly rejected
\item[2] recommended, adapted (with light improvements if necessary)  weakly accepted
\item[3] highly recommended, well adapted,strongly accepted
\item[*] difficult to evaluate with a note (please add a comment under the table)
\end{description}

Where needed additional comments for the given notes are under each table.

\section{Presentation}

This section gives a quick presentation of the approach and the supporting tool.

\begin{description}
\item[Name] means of description: Coloures Petri nets; tool: CPN Tools
\item[Web site] http://cpntools.org
\item[Licence] CPN Tools GUI: GNU GPL version 2; Simulator: GNU GPL version 2 andBSD license; Access/CPN: GNU LGPL version 2.1
\end{description}

\paragraph{Abstract} 

The following evaluation of petri nets and the CPN Tools is in contrast to all other means of description not based on a preliminary model but on earlier formal model of the then current ERTMS/ETCS specification done for Deutsche Bahn AG in 1997. This work developed a full formal system model which was able to provide a
universal means of description for all the development phases. Thereby, an integrated event and data-oriented approach has been used, which is able to describe the different aspects of the  system  on  their  own net  levels.  Coloured Petri nets have been as means of description for this procedure, as they permit universal application, the use of different methods and formal analysis. CPN Tools is a mature tool suite which provides support to edit, check, simulate and analyse nets on all relevant abstraction levels.


\paragraph{Publications} 

\begin{itemize}
\item Van der Aalst, W.M.P. and Stahl, C.: Modeling Business Processes -- A Petri Net-Oriented Approach. The MIT Press, 2011.
\item K. Jensen and Kristensen, L.M.: Coloured Petri Nets -- Modeling and Validation of Concurrent Systems. Springer-Verlag Berlin, 2009.
\item Janhsen, A.; Lemmer, K.; Ptok, B.; Schnieder, E.: Formal Specifications of the European Train Control System. In: Papageorgiou, M.; Pouliezos, A., Hrsg.: Proceedings of 8th IFAC Symposium on Transportation Systems, S. 1215-1220, Chania, Juni 1997. Chania.
\item Meyer zu Hörste, M.; Ptok, B.; Schnieder, E.; Schulz, H.-M.: Modelling and Simulation of the European Train Control System for test case generation. In: Mellitt, B; Hill, R. J.; Allan, J.; Sciutto, G.; Brebbia, C. A., Hrsg.: Proceedings COMPRAIL '98: Computers in Railways VI, Lissabon, S. 649-658, Southampton, Boston, Juli 1998. Sixth international conference on computer aided design, manufacture and operation in the railway and other advanced mass transit systems / Lissabon, WIT Press.
\item Jensen, K.; Kristensen, L.M. and Wells, L.: Coloured Petri Nets and CPN Tools for Modelling and Validation of Concurrent Systems. International Journal on Software Tools for Technology Transfer (STTT)9(3-4), pp. 213-254, 2007. 
\end{itemize}


\section{Main usage of the approach}
\label{main_usage}
The approach has been used to develop a strictly formal model based on an earlier  System  Requirements  Specification version. Therefore the approach uses context, process, scenario and function models to provide visualisation of the system and sub-systen processes as well as the operational processes. 
Petri Nets as a means of description are used in research an in industrial applications used for Process Modeling, Data analysis, Software design and Reliability engineering. Coloured Petri Nets are  High-level Petri Nets which are mainly used to describe, simulate and validated communication between humans and/or computers. As a means of description Coloured and  Hierarchic Petri nets allow to use one uniform means of description for the entire development cycle, starting with  the specification through to  implementation. In addition Petri  nets  provide  the capacity  that allows  different  methods  to  be  used  during  one  single  phase  of  the  development cycle  and also phase-specific methods.
CPN Tools has a graphical editor to model nets and provides various methods to analyse the nets, most importantly a reachability analysis. Resprechtivily the tool supports most applications for Coloured Petri Nets.

According to the figure \ref{fig:main_process}, the approach is recommanded for the following phases (note from 0 to  3) :

\begin{tabular}{|l | c | c | c | c|}
\hline
& \textcolor{green}{Author} & \textcolor{blue}{Assessor 1} & \textcolor{magenta}{Assessor 2} & Total \\
\hline 
System Analysis & 3 & & 3 &  \\
\hline
Sub-system formal design & 3 & & 3 & \\
\hline
Software design & 3 & & 3 & \\
\hline
Software code generation & 1 & & 1 & \\
\hline
\end{tabular}

According to the figure \ref{fig:main_process}, the approach is recommanded for the following types of activities (note from 0 to  3) :

\begin{tabular}{|l | c | c | c | c|}
\hline
& \textcolor{green}{Author} & \textcolor{blue}{Assessor 1} & \textcolor{magenta}{Assessor 2} & Total \\
\hline 
Documentation & 1 & & 1 &  \\
\hline
Modeling & 3 & & 3 &  \\
\hline
Design & 3 & & 2 & \\
\hline
Code generation & 1 & & 1 & \\
\hline
Verification & 3 & & 3 & \\
\hline
Validation & 3 & & 3 & \\
\hline
Safety analyses & 3 & & 2 & \\
\hline
\end{tabular}

\paragraph{Known usages} 

The first goal of the openETCS project, formalising subset 26, has already been handled by this approach as  an earlier version of the ERTMS system requirement specifications has been modeled this way. Thereby it  was  demonstrated  that  during  the  phases  of  system development, covering the system specification through to the final system design, a  model based on Petri nets can be used.

\section{Language}
This section discusses the main element of the language.

Coloured Petri nets have the following main characteristics:

\begin{tabular}{|l | c | c | c | c|}
\hline
& \textcolor{green}{Author} & \textcolor{blue}{Assessor 1} & \textcolor{magenta}{Assessor 2} & Total \\
\hline 
Informal language & & & &  \\
\hline 
Semi-formal language & & & &  \\
\hline
Formal language & x & & &  \\
\hline
Structured language & x & & & \\
\hline
Modular language & x & & & \\
\hline
Textual language & & & & \\
\hline
Mathematical symbols or code & x & & & \\
\hline
Graphical language & x & & & \\
\hline
\end{tabular}

According WP2 requirements, notes are given for the capabilities of Coloured Petri nets (from 0 to 3) :

\begin{tabular}{|l | c | c | c | c|}
\hline
& \textcolor{green}{Author} & \textcolor{blue}{Assessor 1} & \textcolor{magenta}{Assessor 2} & Total \\
\hline
Declarative formalization of properties (D.2.6-X-28) & 3 & & 3 & \\
\hline
Simple formalization of properties (D.2.6-X-28.1) & 3 & & 3 & \\
\hline
Scalability : capability to design large model & 3 & & 3 & \\
\hline
Easily translatable to other languages (D.2.6-X-30) & 2 & & 2 & \\
\hline
Executable directly (D.2.6-X-33) & 3 & & 3 & \\
\hline
Executable after translation to a code (D.2.6-X-33) & 3 & & 3 & \\
(precise if the translation is automatic) & 2 & & 2 & \\
\hline
Simulation, animation (D.2.6-X-33) & 3 & & 3 & \\
\hline
Easily understandable (D.2.6-X-27) & 2 & & 2 & \\
\hline
Expertise level needed (0 High level, 3 few level) & 1 & & 1 & \\
\hline
Standardization (D.2.6-X-29) & 3 & & 3 & \\
\hline
Documented (D.2.6-X-29) & 3 & & 3 & \\
\hline
Extensible language (D.2.6-01-28) & 3 & & 3 & \\
\hline
\end{tabular}


\paragraph{Documentation} Coloured petri nets are standardised as part of the high level petri nets in ISO/IEC 15909 Systems and software engineering - High-level Petri nets. The use of petri nets for the system dependability analysis is standardised in IEC 62551 Analysis techniques for dependability - Petri net modeling. In addition Coloured petri nets and the CPN Tools are introduced and documented in the book \textit{Coloured Petri Nets -- Modeling and Validation of Concurrent Systems} by K. Jensen and L.M. Kristensen. 

\paragraph{Language usage} Basically petri nets are a very powerful completely mathematical defined means of description to graphically model systems in a discrete way. Over the time various extensions have be developed to extend the concept of petri nets to handle a larger amount of system properties and  behaviour. Some of these extensions require export knowledge, so that their models can be difficult to understand for users not familiar with the theory behind petri nets.

\section{System Analysis}
This section discusses the usage of the approach for system analysis.

Acoording WP2 requirements, the approach can be involved for the sub-system requirement specification in the following ways:

\begin{tabular}{|l | c | c | c | c|}
\hline
& \textcolor{green}{Author} & \textcolor{blue}{Assessor 1} & \textcolor{magenta}{Assessor 2} & Total \\
\hline
Independent System functions definition (D.2.6-X-10.2.1)  & 3 & & 3 &  \\
\hline 
System architecture design (D.2.6-X-10.2) & 3 & & 3 &  \\
\hline
System data flow identification (D.2.6-X-10.2.3)  & 3 & & 3 &  \\
\hline
Sub-system focus (D.2.6-X-10.2.4)  & 3 & & 3 &  \\
\hline
System interfaces definition (D.2.6-X-10.2.5)  & 3 & & 3 &  \\
\hline
System requirement allocation (D.2.6-X-10.3)  & 2 & & 2 &  \\
\hline
Traceability with SRS (D.2.6-X-10.5)  & 2 & & 2 &  \\
\hline
Traceability with Safety activities (D.2.6-X-11)  & 2 & & 2 &  \\
\hline
\end{tabular}



\section{Sub-System formal design}
This section discusses the usage of the approach for sub-system formal design.

Petri nets can be used to model both kind of models during this phase: semi-formal models tocover the SSRS (D.2.6-X-12.1) and strictly formal  models to  focus on some functional and safety aspects (D.2.6-X-14).  Obviously petri nets as a strictly  formal means can be used to define the semi-formalmodel.

\subsection{Semi-formal model}

Concerning semi-formal model, the following WP2 requirements are covered:

\begin{tabular}{|l | c | c | c | c|}
\hline
& \textcolor{green}{Author} & \textcolor{blue}{Assessor 1} & \textcolor{magenta}{Assessor 2} & Total \\
\hline 
Consistency to SSRS (D.2.6-X-12.2) & 2 & & * &  \\
\hline
Coverage of SSRS (D.2.6-X-12.2.1)  & 3 & & * &  \\
\hline
Coverage of SSHA (D.2.6-X-12.2.2)  & 3 & & 2 &  \\
\hline
Management of requirement justification (D.2.6-X-12.2.3)  & 1 & & 1 &  \\
\hline
Traceability to  SSRS (D.2.6-X-12.2.5)  & 2 & & 2 &  \\
\hline
Traceability of exported requirements (D.2.6-X-12.2.6)  & 2 & & 2 &  \\
\hline
Simulation or animation (D.2.6-X-13 partial)  & 3 & & 3 &  \\
\hline
Execution (D.2.6-X-13 partial)  & 3 & & 3 &  \\
\hline
Extensible to strictly formal model (D.2.6-X-14.3) & 3 & & 3 &  \\
\hline
Easy to  refine towards strictly formal model (D.2.6-X-14.4) & 3 & & 3 &  \\
\hline
Extensible and modular design (D.2.6-X-15)  & 3 & & 3 &  \\
\hline
Extensible to software architecture and design (D.2.6-X-30)   & 3 & & 3 &  \\
\hline
\end{tabular}

\begin{assessor2}
* Since SSRS is not finished at the moment, those criteria can't be evaluated
\end{assessor2}

Concerning safety properties management, the following WP2 requirements are covere:

\begin{tabular}{|l | c | c | c | c|}
\hline
& \textcolor{green}{Author} & \textcolor{blue}{Assessor 1} & \textcolor{magenta}{Assessor 2} & Total \\
\hline 
Safety function isolation (D.2.6-X-17)  & 3 & & 3 &  \\
\hline 
Safety properties formalisation (D.2.6-X-22)  & 3 & & 3 &  \\
\hline
Logical expression (D.2.6-X-28.2.2)  & 3 & & 3 &  \\
\hline
Timing constraints (D.2.6-X-28.2.3)  & 3 & & 3 &  \\
\hline
Safety properties validation (D.2.6-X-23.2)  & 3 & & 3 &  \\
\hline
Logical properties assertion (D.2.6-X-34)  & 3 & & 3 &  \\
\hline
Check  of assertions (D.2.6-X-34.1)  & 3 & & 3 &  \\
\hline
\end{tabular}

Coloured Petri nets allow to  formalize the following(D.2.6-X-31):

\begin{tabular}{|l | c | c | c | c|}
\hline
& \textcolor{green}{Author} & \textcolor{blue}{Assessor 1} & \textcolor{magenta}{Assessor 2} & Total \\
\hline 
State machines  & 3 & & 3 &  \\
\hline
Time-outs  & 3 & & 3 &  \\
\hline
Truth tables  & 2 & & 2 &  \\
\hline
Arithmetic  & 3 & & 3 &  \\
\hline
Braking curves  & 1 & & 1 &  \\
\hline
Logical statements & 3 & & 3 &  \\
\hline
Message and fields & 3 & & 3 &  \\
\hline
\end{tabular}

\paragraph{Additional comments on semi-formal  model} 
Since the approach has already been used to model the ERTMS/ETCS specification in an earlier version, it is sufficient for the task.
As petri nets are closely related to all means of descriptions based on state machines and automatas the can be compared or translated relatively easy. In addition all verification and validation activities supported by those means of descriptions can be used on petri nets. Although hazard and risk analysis techniques as FTA and FMEA can be translated intro petri nets and petri nets can be used for formal hazard and risk analysis methods.


\subsection{Strictly formal model}

Concerning the strictly formal model, the following WP2 requirements are covered:

\begin{tabular}{|l | c | c | c | c|}
\hline
& \textcolor{green}{Author} & \textcolor{blue}{Assessor 1} & \textcolor{magenta}{Assessor 2} & Total \\
\hline 
Consistency to SFM (D.2.6-X-14.2) & 3 & & 3 &  \\
\hline
Coverage of SSRS (D.2.6-X-14.2)  & 3 & & * &  \\
\hline
Traceability to  SSRS (D.2.6-X-14.3)  & 2 & & * &  \\
\hline
Extensible to software design (D.2.6-X-16)  & 3 & & 3 &  \\
\hline
Safety function isolation (D.2.6-X-17)  & 3 & & 3 &  \\
\hline 
Safety properties formalisation (D.2.6-X-22)  & 3 & & 3 &  \\
\hline
Logical expression (D.2.6-X-28.2.2)  & 3 & & 3 &  \\
\hline
Timing constraints (D.2.6-X-28.2.3)  & 3 & & 3 &  \\
\hline
Safety properties validation (D.2.6-X-23.3)  & 3 & & 3 &  \\
\hline
Logical properties assertion (D.2.6-X-34)  & 3 & & 3 &  \\
\hline
Proof of assertions (D.2.6-X-34.2)  & 3 & & 3 &  \\
\hline
\end{tabular}

\begin{assessor2}
* Since SSRS is not finished at the moment, those criteria can't be evaluated
\end{assessor2}

Coloured petri nets allow to formalize the following(D.2.6-X-32):

\begin{tabular}{|l | c | c | c | c|}
\hline
& \textcolor{green}{Author} & \textcolor{blue}{Assessor 1} & \textcolor{magenta}{Assessor 2} & Total \\
\hline 
State machines  & 3 & & 3 &  \\
\hline
Time-outs  & 3 & & 3 &  \\
\hline
Truth tables  & 2 & & 2 &  \\
\hline
Arithmetic  & 3 & & 3 &  \\
\hline
Braking curves  & 1 & & 1 &  \\
\hline
Logical statements & 3 & & 3 &  \\
\hline
Message and fields & 3 & & 3 &  \\
\hline
\end{tabular}

\paragraph{Additional comments on semi-formal  model} 
Petri nets models can be build directly from the SSRS and then be systematically refined. 
Since petri nets are closely related to all means of descriptions based on state machines and automatas the can be compared or translated relatively easy. In addition all verification and validation activities supported by those means of descriptions can be used on petri nets. Although hazard and risk analysis techniques as FTA and FMEA can be translated intro petri nets and petri nets can be used for formal hazard and risk analysis methods.


\section{Software design}
This section discusses the usage of the approach for software design.

\subsection{Functional design}

The approach allows to  produce a functional software model of the on-board unit in the following ways:

\begin{tabular}{|l | c | c | c | c|}
\hline
& \textcolor{green}{Author} & \textcolor{blue}{Assessor 1} & \textcolor{magenta}{Assessor 2} & Total \\
\hline
Derivation from system semi-formal model  & 3 & & 3 &  \\
\hline 
Software architecture description  & 3 & & 3 &  \\
\hline
Software constraints  & 3 & & 3 &  \\
\hline
Traceability  & 2 & & 2 &  \\
\hline
Executable  & 3 & & 3 &  \\
\hline
\end{tabular}
Since petri nets are a formal means of description usually all refinements require a formal model.

\subsection{SSIL4 design}

The approach allows to  produce in safety a software model by the following ways:

\begin{tabular}{|l | c | c | c | c|}
\hline
& \textcolor{green}{Author} & \textcolor{blue}{Assessor 1} & \textcolor{magenta}{Assessor 2} & Total \\
\hline
Derivation from system semi-formal or strictly formal model  & 3 & & 3 &  \\
\hline 
Software architecture description  & 3 & & 3 &  \\
\hline
Software constraints  & 3 & & 3 &  \\
\hline
Traceability  & 2 & & 2 &  \\
\hline
Executable  & 3 & & 3 &  \\
\hline
Conformance to EN50128 § 7.2  & 3 & & 3 &  \\
\hline
Conformance to EN50128 § 7.3  & 3 & & 3 &  \\
\hline
Conformance to EN50128 § 7.4  & 3 & & 3 &  \\
\hline
\end{tabular}

The following criteria for software architecture are covered by the methodology
(see EN50128 table A.3) :

\begin{tabular}{|l | c | c | c | c|}
\hline
& \textcolor{green}{Author} & \textcolor{blue}{Assessor 1} & \textcolor{magenta}{Assessor 2} & Total \\
\hline
Defensive programming  & & & &  \\
\hline 
Fault detection \& diagnostic  & 2 & & 2 &  \\
\hline
Error detecting code  & 1 & & 1 &  \\
\hline
Failure assertion programming & & & &  \\
\hline
Diverse programming & 1 & & 1 &  \\
\hline
Memorising executed cases & 1 & & 1 &  \\
\hline
Software error effect analysis & 3 & & 3 &  \\
\hline
Fully defined interface & 3 & & 3 &  \\
\hline
Modelling  & 3 & & &  \\
\hline
Structured methodology & 3 & & 3 &  \\
\hline
\end{tabular}

\section{Software code generation}

CPN Tools does not directly generate source code, but the petri nets models are provided in the CPN ML file format, which can be used by other tools to create code

The following criteria for software design and implementation are covered by the methodology:
(see EN50128 table A.4) :

\begin{tabular}{|l | c | c | c | c|}
\hline
& \textcolor{green}{Author} & \textcolor{blue}{Assessor 1} & \textcolor{magenta}{Assessor 2} & Total \\
\hline
Formal methods  & 3 & & 3 &  \\
\hline 
Modeling  & 3 & & 3 &  \\
\hline
Modular approach (mandatory) & 3 & & 3 &  \\
\hline
Components & 3 & & 3 &  \\
\hline
Design and coding standards (mandatory) & 2 & & 2 &  \\
\hline
Strongly typed programming language & 0 & & 0 &  \\
\hline

\end{tabular}



\section{Main usage of the tool}
\label{main_usage}

This section discusses the main usage of the tool.

The following tasks are covered by the tool:


\begin{tabular}{|l | c | c | c | c|}
\hline
& \textcolor{green}{Author} & \textcolor{blue}{Assessor 1} & \textcolor{magenta}{Assessor 2} & Total \\
\hline 
Modelling support & 3 & & 3 &  \\
\hline
Automatic translation  & 1 & & 1 & \\
\hline
Code Generation  & 1 & & 1 & \\
\hline
Model verification & 3 & & 3 & \\
\hline
Test generation & 2 & & 1 & \\
\hline
Simulation, execution, debugging & 3 & & 3 & \\
\hline
Formal proof & 3 & & 3 & \\
\hline
\end{tabular}

\paragraph{Modelling support}
Since petri nets are a primarily graphic means of descrition CPN Tools provide a graphical modelling editor.

\paragraph{Automatic translation and code generation}
CPN Tools does not directly generate source code, but the petri nets models are provided in the CPN ML file format, which can be used by other tools to create code. Petri nets can relatively easy been translated to other means of description based on state machines and automatas. 

\paragraph{Model verification}
Petri nets are mainly verified by generation and analysis of the state space. The tool supports the calculation and drawing of the state space, which is used to verify certain logical and temporal properties of the system.

\paragraph{Test generation}
CPN Tools itself allows simulation of the models. It does not support test generation, but provides interfaces for other tools to do so. Correspondingly, tools like SPENAT can be used to generate and manage all kinds of tests for the nets created with CPN Tools.

\paragraph{Simulation, execution, debugging}
The simulation engine of CPN tools provides a powerful simulation of petri nets and has a number of debugging functions.

\paragraph{Formal proof}
Petri nets are a strictly formal means of description suited for formal proof of behavioural and structural properties. The analysis of the state space can provide proofs for some kinds of properties. Additional model checker can be combined with the tool to provide aditional functionalities.



\section{Use of the tool}


According WP2 requirements, the following notes are given on characteristics of the use of the tool (from 0 to 3):

\begin{tabular}{|l | c | c | c | c|}
\hline
& \textcolor{green}{Author} & \textcolor{blue}{Assessor 1} & \textcolor{magenta}{Assessor 2} & Total \\
\hline 
Open Source (D2.6-X-36) & 3 (almost complete)& & 3 &  \\
\hline 
Portability to operating systems (D2.6-X-37) & 3 & & 3 &  \\
\hline
Cooperation of tools (D2.6-X-38) & 2 & & 2 &  \\
\hline
Robustness (D2.6-X-41) & 2 & & 2 & \\
\hline
Modularity (D2.6-X-41.1) & 3 & & 3 & \\
\hline
Documentation management (D.2.6-X-41.2) & 0 & & 0 & \\
\hline
Distributed software development (D.2.6-X-41.3)  & 2 & & 2 & \\
\hline
Simultaneous multi-users (D.2.6-X-41.4)   & 2 & & * & \\
\hline
Issue tracking (D.2.6-X-41.5) & 0 & & 0 & \\
\hline
Differences between models (D.2.6-X-41.6) & 1 & & 1 & \\
\hline
Version management (D.2.6-X-41.7) & 0 & & 0 & \\
\hline
Concurrent version development (D.2.6-X-41.8) & 0 & & 0 & \\
\hline
Model-based version control (D.2.6-X-41.9) & 1 & & 1 & \\
\hline
Role traceability (D.2.6-X-41.10) & 0 & & 0 & \\
\hline
Safety version traceability (D.2.6-X-41.11) & 0 & & 0 & \\
\hline
Model traceability (D.2.6-01-035) & 2 & & 2 & \\
\hline
Tool chain integration & 2 & & 2 & \\
\hline
Scalability & 3 & & 3 & \\
\hline
\end{tabular}

\section{Certifiability}

This section discusses how the tool can be classified according EN50128 requirements (D.2.6-X-50).


\begin{tabular}{|l | c | c | c | c|}
\hline
& \textcolor{green}{Author} & \textcolor{blue}{Assessor 1} & \textcolor{magenta}{Assessor 2} & Total \\
\hline 
Tool manual (D.2.6-01-42.02) & 3 & & 3 &  \\
\hline
Proof of correctness (D.2.6-01-42.03)   & 3  & & 3 & \\
\hline
Existing industrial  usage  & 2 & & 2 & \\
\hline
Model verification & 3 & & 3 & \\
\hline
Test generation & 1 & & 1 & \\
\hline
Simulation, execution, debugging & 3 & & 3 & \\
\hline
Formal proof & 2 & & 2 & \\
\hline
\end{tabular}


\paragraph{Other elements for tool certification}
This issue can not be specified at this point.


\section{Other comments}
In the context of this approach coloured petri nets present only a very well suited formal means of description. However the method is basically independent of the means of description and can also be applied on other formal means of description. Thereby the proven method to build context models which are then further refined in process, scenario and function models can be successfully used for many formal means of description. 



