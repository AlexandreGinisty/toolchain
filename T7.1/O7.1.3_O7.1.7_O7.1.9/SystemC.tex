\chapter{System C}

\textcolor{red}{
This is a preliminary version of the description and there are still some open questions, also due to unclarities in this template.
}

\begin{description}
\item[\textcolor{green}{Author}] Author of the approaches description  Stefan Rieger (TWT)/ Frank Golatowski (Uni Rostock)
\item[\textcolor{blue}{Assessor 1}] First assessor of the approaches Cecile Braunstein (Uni. Bremen)
\item[\textcolor{magenta}{Assessor 2}] Second assessor of the approaches Silvano DalZilio / LAAS
\end{description}

In the sequel, main text is under the responsibilities of the author.

\begin{author_comment}
Author can add comments using this format at any place.
\end{author_comment}

\begin{assessor1}
First assessor can add comments using this format at any place.
\end{assessor1}

\begin{assessor2}
Second assessor can add comments using this format at any place.
\end{assessor2}

When a note is required, please follow this list :
\begin{description}
\item[0] not recommended, not adapted, rejected
\item[1] weakly recommended, adapted after major improvements, weakly rejected
\item[2] recommended, adapted (with light improvements if necessary)  weakly accepted
\item[3] highly recommended, well adapted,strongly accepted
\item[*] difficult to evaluate with a note (please add a comment under the table)
\end{description}

All the notes can be commented under each table.

\section{Presentation}

This section gives a quick presentation of the approach and the tool.

\begin{description}
\item[Name] SystemC
\item[Web site] \url{http://www.accellera.org/downloads/standards/systemc/about_systemc/}
\item[Licence] SystemC Open Source License
\end{description}

\paragraph{Abstract} 

SystemC is a C++ library providing an event-driven simulation interface suitable for electronic system level design. It enables a system designer to simulate concurrent processes. SystemC processes can communicate in a simulated real-time environment, using channels of different datatypes (all C++ types and user defined types are supported). SystemC supports hardware and software synthesis (with the corresponding tools). SystemC models are executable.


\paragraph{Publications} Short list of publications on the approach (5 max)

\begin{itemize}
\item D. C. Black, SystemC: From the ground up. Springer, 2010.
\item IEEE 1666 Standard SystemC Language Reference Manual, \url{http://standards.ieee.org/getieee/1666/}
\item The ITEA MARTES Project, from UML to SystemC, \url{http://www.martes-itea.org/}
\item J. Bhasker, A SystemC Primer, Second Edition, Star Galaxy Publishing, 2004
\item F. Ghenassia (Editor), Transaction-Level Modeling with SystemC: TLM Concepts and Applications for Embedded Systems, Springer 2006
\end{itemize}

\section{Main usage of the approach}
\label{main_usage}

SystemC is suitable for system level design at various abstraction levels (from high level down to individual hardware components) and can thus be employed to build a full system model. Due to its modular design and abstraction priciples sub-components and a lower abstraction level of the model can be considered as ``black boxes''. SystemC models can be executed and simulated allowing for testing of the entire model or individual components.

According to the figure \ref{fig:main_process}, for which phases do you recommend the approach (give a note from 0 to  3) :

\begin{tabular}{|l | c | c | c | c|}
\hline
& \textcolor{green}{Author} & \textcolor{blue}{Assessor 1} & \textcolor{magenta}{Assessor 2} & Total \\
\hline 
System Analysis &1 & & &  \\
\hline
Sub-system formal design &3 & & & \\
\hline
Software design &3 & & & \\
\hline
Software code generation &2 & & & \\
\hline
\end{tabular}

According to the figure \ref{fig:main_process}, for which type of activities do you recommend the approach (give a note from 0 to  3) :

\begin{tabular}{|l | c | c | c | c|}
\hline
& \textcolor{green}{Author} & \textcolor{blue}{Assessor 1} & \textcolor{magenta}{Assessor 2} & Total \\
\hline 
Documentation &0 & & &  \\
\hline
Modeling &3 & & &  \\
\hline
Design &3 & & & \\
\hline
Code generation &3 & & & \\
\hline
Verification &2 & & & \\
\hline
Validation &2 & & & \\
\hline
Safety analyses &2 & & & \\
\hline
\end{tabular}

\paragraph{Known usages} Have you some examples of usage of this approach to  compare with the OpenETCS objectives ?

SystemC has been applied, among others, in the following areas:

\begin{itemize}
	\item Communication technology
	\item Hardware design and simulation
	\item Hardware and software synthesis
	\item Sensor circuits
	\item Automotive
	\item Aerospace industry
\end{itemize}

SystemC is widely employed in industry. Among the members of the Accellera Systems Initiative responsible for SystemC are the following organisations:

AMD, ARM, Cadence, Intel, NXP, Qualcomm, Synopsys, Texas Instruments, Altera, Boeing, Cisco, Ericsson, Fraunhofer IIS, IBM, NEC, nVidia, Xilinx

Vendors supporting SystemC (according to \href{http://en.wikipedia.org/wiki/SystemC}{Wikipedia}):

Aldec, AutoESL, Cadence Design Systems, HCL Technologies, Calypto, CircuitSutra, CoFluent Design, CoSynth Synthesizer, CoWare, Forte Design Systems, Mentor Graphics, OVPsim, NEC CyberWorkBench, Imperas, Synopsys, SystemCrafter, JEDA Technologies, HIFSuite, Dynalith Systems, VWorks


\section{Language}
This section discusses the main element of the language.

Which are the main characteristics of the language :

\begin{tabular}{|l | c | c | c | c|}
\hline
& \textcolor{green}{Author} & \textcolor{blue}{Assessor 1} & \textcolor{magenta}{Assessor 2} & Total \\
\hline 
Informal language &0 & & &  \\
\hline 
Semi-formal language &3 & & &  \\
\hline
Formal language &2 & & &  \\
\hline
Structured language &3 & & & \\
\hline
Modular language &3 & & & \\
\hline
Textual language &3 & & & \\
\hline
Mathematical symbols or code &3 & & & \\
\hline
Graphical language &* & & & \\
\hline
\end{tabular}

\begin{author_comment}
* Not graphical, but we are investigating SystemC and UML/SysML integration, the \href{http://www.martes-idea.org/}{ITEA MARTES Project} is addressing this aspect
\end{author_comment}

According WP2 requirements, give a note for the capabilities of the language (from 0 to 3) :

\begin{tabular}{|l | c | c | c | c|}
\hline
& \textcolor{green}{Author} & \textcolor{blue}{Assessor 1} & \textcolor{magenta}{Assessor 2} & Total \\
\hline
Declarative formalization of properties (D.2.6-X-28) &1 & & & \\
\hline
Simple formalization of properties (D.2.6-X-28.1) &1 & & & \\
\hline
Scalability : capability to design large model &3 & & & \\
\hline
Easily translatable to other languages (D.2.6-X-30) &2 & & & \\
\hline
Executable directly (D.2.6-X-33) &3 & & & \\
\hline
Executable after translation to a code (D.2.6-X-33) &3 & & & \\
(precise if the translation is automatic) & & & & \\
\hline
Simulation, animation (D.2.6-X-33) &3 & & & \\
\hline
Easily understandable (D.2.6-X-27) &2 & & & \\
\hline
Expertise level needed (0 High level, 3 few level) &1 & & & \\
\hline
Standardization (D.2.6-X-29) &3 & & & \\
\hline
Documented (D.2.6-X-29) &3 & & & \\
\hline
Extensible language (D.2.6-01-28) &3 & & & \\
\hline
\end{tabular}


\paragraph{Documentation} Describe how the language is documented, the existing guidelines, coding rules, standardization...

There is an IEEE Standard covering SystemC, an official specification from the Accellera Initiative and a plethora of third party literature and tutorials.

\paragraph{Language usage} Describe the possible restriction on the language

\begin{itemize}
\item As the language is based on C++ and thus inherits its expressivity there might be problems in static analysis if the models use the power of the language in an unrestricted manner.
\item The language is text-based and not graphical. However, there are approaches of integrating SystemC and UML/SysML. We are currently investigating in this issue.
\end{itemize}


\section{System Analysis}
This section discusses the usage of the approach for system analysis.
It can be skipped depending the results of \ref{main_usage}.

Acoording WP2 requirements, how the approach can be involved for the sub-system requirement specification ?

\begin{tabular}{|l | c | c | c | c|}
\hline
& \textcolor{green}{Author} & \textcolor{blue}{Assessor 1} & \textcolor{magenta}{Assessor 2} & Total \\
\hline
Independent System functions definition (D.2.6-X-10.2.1)  &3 & & &  \\
\hline 
System architecture design (D.2.6-X-10.2) &3 & & &  \\
\hline
System data flow identification (D.2.6-X-10.2.3)  &3 & & &  \\
\hline
Sub-system focus (D.2.6-X-10.2.4)  &3 & & &  \\
\hline
System interfaces definition (D.2.6-X-10.2.5)  &3 & & &  \\
\hline
System requirement allocation (D.2.6-X-10.3)  &2* & & &  \\
\hline
Traceability with SRS (D.2.6-X-10.5)  &** & & &  \\
\hline
Traceability with Safety activities (D.2.6-X-11)  &2* & & &  \\
\hline
\end{tabular}

\begin{author_comment}
* Can possibly be covered by an associated SysML model. In addition, standardised, machine readable comments in the code could be used.\\
** This is not the scope of SystemC
\end{author_comment}

\section{Sub-System formal design}
This section discusses the usage of the approach for sub-system formal design.
It can be skipped depending the results of \ref{main_usage}.

Two kinds of model can be planned during this phase: semi-formal models to  cover the SSRS (D.2.6-X-12.1) and strictly formal  models to  focuss on some functional and safety aspects (D.2.6-X-14).  Obviously some strictly  formal means can be used to define the semi-formal  model.

\subsection{Semi-formal model}

Concerning semi-formal model, how the WP2 requirements are covered ?

\begin{tabular}{|l | c | c | c | c|}
\hline
& \textcolor{green}{Author} & \textcolor{blue}{Assessor 1} & \textcolor{magenta}{Assessor 2} & Total \\
\hline 
Consistency to SSRS (D.2.6-X-12.2) &3 & & &  \\
\hline
Coverage of SSRS (D.2.6-X-12.2.1)  &* & & &  \\
\hline
Coverage of SSHA (D.2.6-X-12.2.2)  &* & & &  \\
\hline
Management of requirement justification (D.2.6-X-12.2.3)  &2 & & &  \\
\hline
Traceability to  SSRS (D.2.6-X-12.2.5)  &** & & &  \\
\hline
Traceability of exported requirements (D.2.6-X-12.2.6)  &*** & & &  \\
\hline
Simulation or animation (D.2.6-X-13 partial)  &3 & & &  \\
\hline
Execution (D.2.6-X-13 partial)  &3 & & &  \\
\hline
Extensible to strictly formal model (D.2.6-X-14.3) &2 & & &  \\
\hline
Easy to  refine towards strictly formal model (D.2.6-X-14.4) &2 & & &  \\
\hline
Extensible and modular design (D.2.6-X-15)  &3 & & &  \\
\hline
Extensible to software architecture and design (D.2.6-X-30)   &3 & & &  \\
\hline
\end{tabular}

\begin{author_comment}
* The coverage has to be achieved by the \textbf{model}, not by the language or tool and should be removed from the table.\\
** See table above\\
*** What are ``exported requirements''?\\
\end{author_comment}

Concerning safety properties management, how the WP2 requirements are covered ?

\begin{tabular}{|l | c | c | c | c|}
\hline
& \textcolor{green}{Author} & \textcolor{blue}{Assessor 1} & \textcolor{magenta}{Assessor 2} & Total \\
\hline 
Safety function isolation (D.2.6-X-17)  &* & & &  \\
\hline 
Safety properties formalisation (D.2.6-X-22)  &2 & & &  \\
\hline
Logical expression (D.2.6-X-28.2.2)  &3 & & &  \\
\hline
Timing constraints (D.2.6-X-28.2.3)  &3 & & &  \\
\hline
Safety properties validation (D.2.6-X-23.2)  &3 & & &  \\
\hline
Logical properties assertion (D.2.6-X-34)  &3 & & &  \\
\hline
Check  of assertions (D.2.6-X-34.1)  &3 & & &  \\
\hline
\end{tabular}

\begin{author_comment}
* Item not clear to me, should be a requirement for the actual implementation, not a model
\end{author_comment}

Does the language allow to  formalize (D.2.6-X-31):

\begin{tabular}{|l | c | c | c | c|}
\hline
& \textcolor{green}{Author} & \textcolor{blue}{Assessor 1} & \textcolor{magenta}{Assessor 2} & Total \\
\hline 
State machines  &3 & & &  \\
\hline
Time-outs  &3 & & &  \\
\hline
Truth tables  &3 & & &  \\
\hline
Arithmetic  &3 & & &  \\
\hline
Braking curves  &3 & & &  \\
\hline
Logical statements &3 & & &  \\
\hline
Message and fields &3 & & &  \\
\hline
\end{tabular}

\paragraph{Additional comments on semi-formal  model} Do you think your semi-formal  model is sufficient to cover a safe design of the on-board unit until code generation ?
All comments on links to  other models, validation and verification activities are welcomed.

\subsection{Strictly formal model}

\begin{author_comment}
Not filled, since we do not consider SystemC to be a strictly formal modelling language, as it has no mathematically formalized sematics. Fully formal models should also support ``really'' formal verification (not only testing) which requires additional work here. However, there are many approaches in the literature to, e.g., apply model checking to SystemC models.
\end{author_comment}

Concerning strictly formal model, how the WP2 requirements are covered ?

\begin{tabular}{|l | c | c | c | c|}
\hline
& \textcolor{green}{Author} & \textcolor{blue}{Assessor 1} & \textcolor{magenta}{Assessor 2} & Total \\
\hline 
Consistency to SFM (D.2.6-X-14.2) & & & &  \\
\hline
Coverage of SSRS (D.2.6-X-14.2)  & & & &  \\
\hline
Traceability to  SSRS (D.2.6-X-14.3)  & & & &  \\
\hline
Extensible to software design (D.2.6-X-16)  & & & &  \\
\hline
Safety function isolation (D.2.6-X-17)  & & & &  \\
\hline 
Safety properties formalisation (D.2.6-X-22)  & & & &  \\
\hline
Logical expression (D.2.6-X-28.2.2)  & & & &  \\
\hline
Timing constraints (D.2.6-X-28.2.3)  & & & &  \\
\hline
Safety properties validation (D.2.6-X-23.3)  & & & &  \\
\hline
Logical properties assertion (D.2.6-X-34)  & & & &  \\
\hline
Proof of assertions (D.2.6-X-34.2)  & & & &  \\
\hline
\end{tabular}

Does the language allow to  formalize (D.2.6-X-32):

\begin{tabular}{|l | c | c | c | c|}
\hline
& \textcolor{green}{Author} & \textcolor{blue}{Assessor 1} & \textcolor{magenta}{Assessor 2} & Total \\
\hline 
State machines  & & & &  \\
\hline
Time-outs  & & & &  \\
\hline
Truth tables  & & & &  \\
\hline
Arithmetic  & & & &  \\
\hline
Braking curves  & & & &  \\
\hline
Logical statements & & & &  \\
\hline
Message and fields & & & &  \\
\hline
\end{tabular}

\paragraph{Additional comments on semi-formal  model} Do you think your strictly formal  model can be directly defined from the SSRS ?
All comments on links to  other models, validation and verification activities are welcomed.


\section{Software design}
This section discusses the usage of the approach for software design.
It can be skipped depending the results of \ref{main_usage}.

\begin{author_comment}
SystemC allows system, software and hardware design and is thus suitable.
\end{author_comment}

\subsection{Functional design}

How the approach allows to  produce a functional software model of the on-board unit ?

\begin{tabular}{|l | c | c | c | c|}
\hline
& \textcolor{green}{Author} & \textcolor{blue}{Assessor 1} & \textcolor{magenta}{Assessor 2} & Total \\
\hline
Derivation from system semi-formal model  &* & & &  \\
\hline 
Software architecture description  &3 & & &  \\
\hline
Software constraints  &3 & & &  \\
\hline
Traceability  &2** & & &  \\
\hline
Executable  &3 & & &  \\
\hline
\end{tabular}

\begin{author_comment}
* Item unclear to me\\
** Can possibly be covered by an associated SysML model. In addition, standardised, machine readable comments in the code could be used.\\
\end{author_comment}

\subsection{SSIL4 design}

How the approach allows to  produce in safety a software model ?

\begin{tabular}{|l | c | c | c | c|}
\hline
& \textcolor{green}{Author} & \textcolor{blue}{Assessor 1} & \textcolor{magenta}{Assessor 2} & Total \\
\hline
Derivation from system semi-formal or strictly formal model  &* & & &  \\
\hline 
Software architecture description  &3 & & &  \\
\hline
Software constraints  &3 & & &  \\
\hline
Traceability  &2** & & &  \\
\hline
Executable  &3 & & &  \\
\hline
Conformance to EN50128 § 7.2  &*** & & &  \\
\hline
Conformance to EN50128 § 7.3  &*** & & &  \\
\hline
Conformance to EN50128 § 7.4  &*** & & &  \\
\hline
\end{tabular}

\begin{author_comment}
* Item unclear to me\\
** Can possibly be covered by an associated SysML model. In addition, standardised, machine readable comments in the code could be used.\\
*** No idea, why don't you cite these items?
\end{author_comment}

Which criteria for software architecture are covered by the methodology
(see EN50128 table A.3) :

\begin{tabular}{|l | c | c | c | c|}
\hline
& \textcolor{green}{Author} & \textcolor{blue}{Assessor 1} & \textcolor{magenta}{Assessor 2} & Total \\
\hline
Defensive programming  &* & & &  \\
\hline 
Fault detection \& diagnostic  &2 & & &  \\
\hline
Error detecting code  &3 & & &  \\
\hline
Failure assertion programming &3 & & &  \\
\hline
Diverse programming &* & & &  \\
\hline
Memorising executed cases &3 & & &  \\
\hline
Software error effect analysis &* & & &  \\
\hline
Fully defined interface &3 & & &  \\
\hline
Modelling  &3 & & &  \\
\hline
Structured methodology &3 & & &  \\
\hline
\end{tabular}

\begin{author_comment}
 * SystemC is a language and no methodology. These methodologies can be applied for most languages.
\end{author_comment}

\section{Software code generation}
This section discusses the usage of the approach for software code generation.
It can be skipped depending the results of \ref{main_usage}.

Which criteria for software design and implementation are covered by the methodology
(see EN50128 table A.4) :

\begin{tabular}{|l | c | c | c | c|}
\hline
& \textcolor{green}{Author} & \textcolor{blue}{Assessor 1} & \textcolor{magenta}{Assessor 2} & Total \\
\hline
Formal methods  &0* & & &  \\
\hline 
Modeling  &3 & & &  \\
\hline
Modular approach (mandatory) &3 & & &  \\
\hline
Components &3 & & &  \\
\hline
Design and coding standards (mandatory) &** & & &  \\
\hline
Strongly typed programming language &2 & & &  \\
\hline

\end{tabular}

\begin{author_comment}
 * Not integrated in the language, requires external tools/methods (there's a plethora of approaches in the literature)\\
 ** Have to be stated by the project
\end{author_comment}

\section{Main usage of the tool}
\label{main_usage}

This section discusses the main usage of the tool.

Which task are covered by the tool ?


\begin{tabular}{|l | c | c | c | c|}
\hline
& \textcolor{green}{Author} & \textcolor{blue}{Assessor 1} & \textcolor{magenta}{Assessor 2} & Total \\
\hline 
Modelling support &3 & & &  \\
\hline
Automatic translation  &3 & & & \\
\hline
Code Generation  &2* & & & \\
\hline
Model verification &2 & & & \\
\hline
Test generation &2 & & & \\
\hline
Simulation, execution, debugging &3 & & & \\
\hline
Formal proof &0 & & & \\
\hline
\end{tabular}

\begin{author_comment}
 * The model is itself executable
\end{author_comment}

\paragraph{Modelling support}
Does the tool provide a  textual or a graphical editor ?

It is a textual language. We are investigating in a SysML/UML integration, see above.

\paragraph{Automatic translation and code generation}
Which translation or code generation is supported by the tool ?

The model is itself executable with an integrated simulation environment, but there is a variety of tool providers for software synthesis (see above)

\paragraph{Model verification}
Which verification on models are provided by the tool?

Simulation, Testing

\paragraph{Test generation}
Does the tool allow to generate tests ? For  which purpose ?

There are extensions that support generating random tests with constraints.

\paragraph{Simulation, execution, debugging}
Does the tool allow to simulate or to debbug step by step a model or a code ?

Simulation is supported, debugging can be done by any C++ debugger.

\paragraph{Formal proof}
Does the tool allow formal proof ?  How ?

No, only by means of external tools


\section{Use of the tool}


According WP2 requirements, give a note for characteristics of the use of the tool (from 0 to 3) :

\begin{tabular}{|l | c | c | c | c|}
\hline
& \textcolor{green}{Author} & \textcolor{blue}{Assessor 1} & \textcolor{magenta}{Assessor 2} & Total \\
\hline 
Open Source (D2.6-X-36) &3 & & &  \\
\hline 
Portability to operating systems (D2.6-X-37) &3 & & &  \\
\hline
Cooperation of tools (D2.6-X-38) &* & & &  \\
\hline
Robustness (D2.6-X-41) &3 & & & \\
\hline
Modularity (D2.6-X-41.1) &3 & & & \\
\hline
Documentation management (D.2.6-X-41.2) &0 & & & \\
\hline
Distributed software development (D.2.6-X-41.3)  &3** & & & \\
\hline
Simultaneous multi-users (D.2.6-X-41.4)   &3** & & & \\
\hline
Issue tracking (D.2.6-X-41.5) &0 & & & \\
\hline
Differences between models (D.2.6-X-41.6) &3** & & & \\
\hline
Version management (D.2.6-X-41.7) &3** & & & \\
\hline
Concurrent version development (D.2.6-X-41.8) &3** & & & \\
\hline
Model-based version control (D.2.6-X-41.9) &*** & & & \\
\hline
Role traceability (D.2.6-X-41.10) &* & & & \\
\hline
Safety version traceability (D.2.6-X-41.11) &* & & & \\
\hline
Model traceability (D.2.6-01-035) &**** & & & \\
\hline
Tool chain integration &2***** & & & \\
\hline
Scalability &3 & & & \\
\hline
\end{tabular}

\begin{author_comment}
  * Unclear to me\\
 ** By means of versioning systems such as Git or SVN\\
*** For SystemC text-based version control is equivalent to model-based version control.\\
**** Can possibly be covered by an associated SysML model. In addition, standardised, machine readable comments in the code could be used.\\
***** Tool chain integration can be achieved at different levels. E.g., SystemC can be the target language from graphical, higher-level languages (e.g., SysML). SystemC models are executable and thus code generation is possibly no issue if we want to obtain just an executable model but no real code running on the target platform (which is out of scope for openETCS).
\end{author_comment}

\section{Certifiability}

This section discusses how the tool can be classified according EN50128 requirements (D.2.6-X-50).

\begin{author_comment}
  I have no clue here.
\end{author_comment}

\begin{tabular}{|l | c | c | c | c|}
\hline
& \textcolor{green}{Author} & \textcolor{blue}{Assessor 1} & \textcolor{magenta}{Assessor 2} & Total \\
\hline 
Tool manual (D.2.6-01-42.02) & & & &  \\
\hline
Proof of correctness (D.2.6-01-42.03)   & & & & \\
\hline
Existing industrial  usage  & & & & \\
\hline
Model verification & & & & \\
\hline
Test generation & & & & \\
\hline
Simulation, execution, debugging & & & & \\
\hline
Formal proof & & & & \\
\hline
\end{tabular}

\paragraph{Other elements for tool certification}

\section{Other comments}
Please to  give free comments on the approach.



