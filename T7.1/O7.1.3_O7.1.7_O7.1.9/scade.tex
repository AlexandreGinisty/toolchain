\chapter{SCADE}

\begin{description}
\item[\textcolor{green}{Author}] Author of the approaches description  Uwe Steinke (Siemens)
\item[\textcolor{blue}{Assessor 1}] First assessor of the approaches David Mentré (MERCE)
\item[\textcolor{magenta}{Assessor 2}] Second assessor of the approaches Cécile Braunstein (Uni. Bremen)
\end{description}

In the sequel, main text is under the responsibilities of the author.

\begin{author_comment}
Author can add comments using this format at any place.
\end{author_comment}

\begin{assessor1}
First assessor can add comments using this format at any place.
\end{assessor1}

\begin{assessor2}
Second assessor can add comments using this format at any place.
\end{assessor2}

When a note is required, please follow this list :
\begin{description}
\item[0] not recommended, not adapted, rejected
\item[1] weakly recommended, adapted after major improvements, weakly rejected
\item[2] recommended, adapted (with light improvements if necessary)  weakly accepted
\item[3] highly recommended, well adapted,strongly accepted
\item[*] difficult to evaluate with a note (please add a comment under the table)
\end{description}

All the notes can be commented under each table.

\section{Presentation}

This section gives a quick presentation of the approach and the tool.

\begin{description}
\item[Name] \todo{Name of the approach and the tool}
\item[Web site] \todo{if available, how to  find information}
\item[Licence] \todo{Kind of licence}
\end{description}

\paragraph{Abstract} Short abstract on the approach and tool (10 lines max)

\paragraph{Publications} Short list of publications on the approach (5 max)


\section{Main usage of the approach}
\label{main_usage}
This section discusses the main usage of the approach.

According to the figure \ref{fig:main_process}, for which phases do you recommend the approach (give a note from 0 to  3) :

\begin{tabular}{|l | c | c | c | c|}
\hline
& \textcolor{green}{Author} & \textcolor{blue}{Assessor 1} & \textcolor{magenta}{Assessor 2} & Total \\
\hline 
System Analysis & & & &  \\
\hline
Sub-system formal design & & & & \\
\hline
Software design & & & & \\
\hline
Software code generation & & & & \\
\hline
\end{tabular}

According to the figure \ref{fig:main_process}, for which type of activities do you recommend the approach (give a note from 0 to  3) :

\begin{tabular}{|l | c | c | c | c|}
\hline
& \textcolor{green}{Author} & \textcolor{blue}{Assessor 1} & \textcolor{magenta}{Assessor 2} & Total \\
\hline 
Documentation & & & &  \\
\hline
Modeling & & & &  \\
\hline
Design & & & & \\
\hline
Code generation & & & & \\
\hline
Verification & & & & \\
\hline
Validation & & & & \\
\hline
Safety analyses & & & & \\
\hline
\end{tabular}

\paragraph{Known usages} Have you some examples of usage of this approach to  compare with the OpenETCS objectives ?

\section{Language}
This section discusses the main element of the language.

Which are the main characteristics of the language :

\begin{tabular}{|l | c | c | c | c|}
\hline
& \textcolor{green}{Author} & \textcolor{blue}{Assessor 1} & \textcolor{magenta}{Assessor 2} & Total \\
\hline 
Informal language & & & &  \\
\hline 
Semi-formal language & & & &  \\
\hline
Formal language & & & &  \\
\hline
Structured language & & & & \\
\hline
Modular language & & & & \\
\hline
Textual language & & & & \\
\hline
Mathematical symbols or code & & & & \\
\hline
Graphical language & & & & \\
\hline
\end{tabular}

According WP2 requirements, give a note for the capabilities of the language (from 0 to 3) :

\begin{tabular}{|l | c | c | c | c|}
\hline
& \textcolor{green}{Author} & \textcolor{blue}{Assessor 1} & \textcolor{magenta}{Assessor 2} & Total \\
\hline
Declarative formalization of properties (D.2.6-X-28) & & & & \\
\hline
Simple formalization of properties (D.2.6-X-28.1) & & & & \\
\hline
Scalability : capability to design large model & & & & \\
\hline
Easily translatable to other languages (D.2.6-X-30) & & & & \\
\hline
Executable directly (D.2.6-X-33) & & & & \\
\hline
Executable after translation to a code (D.2.6-X-33) & & & & \\
(precise if the translation is automatic) & & & & \\
\hline
Simulation, animation (D.2.6-X-33) & & & & \\
\hline
Easily understandable (D.2.6-X-27) & & & & \\
\hline
Expertise level needed (0 High level, 3 few level) & & & & \\
\hline
Standardization (D.2.6-X-29) & & & & \\
\hline
Documented (D.2.6-X-29) & & & & \\
\hline
Extensible language (D.2.6-01-28) & & & & \\
\hline
\end{tabular}


\paragraph{Documentation} Describe how the language is documented, the existing guidelines, coding rules, standardization...

\paragraph{Language usage} Describe the possible restriction on the language

\section{System Analysis}
This section discusses the usage of the approach for system analysis.
It can be skipped depending the results of \ref{main_usage}.

Acoording WP2 requirements, how the approach can be involved for the sub-system requirement specification ?

\begin{tabular}{|l | c | c | c | c|}
\hline
& \textcolor{green}{Author} & \textcolor{blue}{Assessor 1} & \textcolor{magenta}{Assessor 2} & Total \\
\hline
Independent System functions definition (D.2.6-X-10.2.1)  & & & &  \\
\hline 
System architecture design (D.2.6-X-10.2) & & & &  \\
\hline
System data flow identification (D.2.6-X-10.2.3)  & & & &  \\
\hline
Sub-system focus (D.2.6-X-10.2.4)  & & & &  \\
\hline
System interfaces definition (D.2.6-X-10.2.5)  & & & &  \\
\hline
System requirement allocation (D.2.6-X-10.3)  & & & &  \\
\hline
Traceability with SRS (D.2.6-X-10.5)  & & & &  \\
\hline
Traceability with Safety activities (D.2.6-X-11)  & & & &  \\
\hline
\end{tabular}



\section{Sub-System formal design}
This section discusses the usage of the approach for sub-system formal design.
It can be skipped depending the results of \ref{main_usage}.

Two kinds of model can be planned during this phase: semi-formal models to  cover the SSRS (D.2.6-X-12.1) and strictly formal  models to  focuss on some functional and safety aspects (D.2.6-X-14).  Obviously some strictly  formal means can be used to define the semi-formal  model.

\subsection{Semi-formal model}

Concerning semi-formal model, how the WP2 requirements are covered ?

\begin{tabular}{|l | c | c | c | c|}
\hline
& \textcolor{green}{Author} & \textcolor{blue}{Assessor 1} & \textcolor{magenta}{Assessor 2} & Total \\
\hline 
Consistency to SSRS (D.2.6-X-12.2) & & & &  \\
\hline
Coverage of SSRS (D.2.6-X-12.2.1)  & & & &  \\
\hline
Coverage of SSHA (D.2.6-X-12.2.2)  & & & &  \\
\hline
Management of requirement justification (D.2.6-X-12.2.3)  & & & &  \\
\hline
Traceability to  SSRS (D.2.6-X-12.2.5)  & & & &  \\
\hline
Traceability of exported requirements (D.2.6-X-12.2.6)  & & & &  \\
\hline
Simulation or animation (D.2.6-X-13 partial)  & & & &  \\
\hline
Execution (D.2.6-X-13 partial)  & & & &  \\
\hline
Extensible to strictly formal model (D.2.6-X-14.3) & & & &  \\
\hline
Easy to  refine towards strictly formal model (D.2.6-X-14.4) & & & &  \\
\hline
Extensible and modular design (D.2.6-X-15)  & & & &  \\
\hline
Extensible to software architecture and design (D.2.6-X-30)   & & & &  \\
\hline
\end{tabular}

Concerning safety properties management, how the WP2 requirements are covered ?

\begin{tabular}{|l | c | c | c | c|}
\hline
& \textcolor{green}{Author} & \textcolor{blue}{Assessor 1} & \textcolor{magenta}{Assessor 2} & Total \\
\hline 
Safety function isolation (D.2.6-X-17)  & & & &  \\
\hline 
Safety properties formalisation (D.2.6-X-22)  & & & &  \\
\hline
Logical expression (D.2.6-X-28.2.2)  & & & &  \\
\hline
Timing constraints (D.2.6-X-28.2.3)  & & & &  \\
\hline
Safety properties validation (D.2.6-X-23.2)  & & & &  \\
\hline
Logical properties assertion (D.2.6-X-34)  & & & &  \\
\hline
Check  of assertions (D.2.6-X-34.1)  & & & &  \\
\hline
\end{tabular}

Does the language allow to  formalize (D.2.6-X-31):

\begin{tabular}{|l | c | c | c | c|}
\hline
& \textcolor{green}{Author} & \textcolor{blue}{Assessor 1} & \textcolor{magenta}{Assessor 2} & Total \\
\hline 
State machines  & & & &  \\
\hline
Time-outs  & & & &  \\
\hline
Truth tables  & & & &  \\
\hline
Arithmetic  & & & &  \\
\hline
Braking curves  & & & &  \\
\hline
Logical statements & & & &  \\
\hline
Message and fields & & & &  \\
\hline
\end{tabular}

\paragraph{Additional comments on semi-formal  model} Do you think your semi-formal  model is sufficient to cover a safe design of the on-board unit until code generation ?
All comments on links to  other models, validation and verification activities are welcomed.

\subsection{Strictly formal model}

Concerning strictly formal model, how the WP2 requirements are covered ?

\begin{tabular}{|l | c | c | c | c|}
\hline
& \textcolor{green}{Author} & \textcolor{blue}{Assessor 1} & \textcolor{magenta}{Assessor 2} & Total \\
\hline 
Consistency to SFM (D.2.6-X-14.2) & & & &  \\
\hline
Coverage of SSRS (D.2.6-X-14.2)  & & & &  \\
\hline
Traceability to  SSRS (D.2.6-X-14.3)  & & & &  \\
\hline
Extensible to software design (D.2.6-X-16)  & & & &  \\
\hline
Safety function isolation (D.2.6-X-17)  & & & &  \\
\hline 
Safety properties formalisation (D.2.6-X-22)  & & & &  \\
\hline
Logical expression (D.2.6-X-28.2.2)  & & & &  \\
\hline
Timing constraints (D.2.6-X-28.2.3)  & & & &  \\
\hline
Safety properties validation (D.2.6-X-23.3)  & & & &  \\
\hline
Logical properties assertion (D.2.6-X-34)  & & & &  \\
\hline
Proof of assertions (D.2.6-X-34.2)  & & & &  \\
\hline
\end{tabular}

Does the language allow to  formalize (D.2.6-X-32):

\begin{tabular}{|l | c | c | c | c|}
\hline
& \textcolor{green}{Author} & \textcolor{blue}{Assessor 1} & \textcolor{magenta}{Assessor 2} & Total \\
\hline 
State machines  & & & &  \\
\hline
Time-outs  & & & &  \\
\hline
Truth tables  & & & &  \\
\hline
Arithmetic  & & & &  \\
\hline
Braking curves  & & & &  \\
\hline
Logical statements & & & &  \\
\hline
Message and fields & & & &  \\
\hline
\end{tabular}

\paragraph{Additional comments on semi-formal  model} Do you think your strictly formal  model can be directly defined from the SSRS ?
All comments on links to  other models, validation and verification activities are welcomed.


\section{Software design}
This section discusses the usage of the approach for software design.
It can be skipped depending the results of \ref{main_usage}.

\subsection{Functional design}

How the approach allows to  produce a functional software model of the on-board unit ?

\begin{tabular}{|l | c | c | c | c|}
\hline
& \textcolor{green}{Author} & \textcolor{blue}{Assessor 1} & \textcolor{magenta}{Assessor 2} & Total \\
\hline
Derivation from system semi-formal model  & & & &  \\
\hline 
Software architecture description  & & & &  \\
\hline
Software constraints  & & & &  \\
\hline
Traceability  & & & &  \\
\hline
Executable  & & & &  \\
\hline
\end{tabular}

\subsection{SSIL4 design}

How the approach allows to  produce in safety a software model ?

\begin{tabular}{|l | c | c | c | c|}
\hline
& \textcolor{green}{Author} & \textcolor{blue}{Assessor 1} & \textcolor{magenta}{Assessor 2} & Total \\
\hline
Derivation from system semi-formal or strictly formal model  & & & &  \\
\hline 
Software architecture description  & & & &  \\
\hline
Software constraints  & & & &  \\
\hline
Traceability  & & & &  \\
\hline
Executable  & & & &  \\
\hline
Conformance to EN50128 § 7.2  & & & &  \\
\hline
Conformance to EN50128 § 7.3  & & & &  \\
\hline
Conformance to EN50128 § 7.4  & & & &  \\
\hline
\end{tabular}

Which criteria for software architecture are covered by the methodology
(see EN50128 table A.3) :

\begin{tabular}{|l | c | c | c | c|}
\hline
& \textcolor{green}{Author} & \textcolor{blue}{Assessor 1} & \textcolor{magenta}{Assessor 2} & Total \\
\hline
Defensive programming  & & & &  \\
\hline 
Fault detection \& diagnostic  & & & &  \\
\hline
Error detecting code  & & & &  \\
\hline
Failure assertion programming & & & &  \\
\hline
Diverse programming & & & &  \\
\hline
Memorising executed cases & & & &  \\
\hline
Software error effect analysis & & & &  \\
\hline
Fully defined interface & & & &  \\
\hline
Modelling  & & & &  \\
\hline
Structured methodology & & & &  \\
\hline
\end{tabular}

\section{Software code generation}
This section discusses the usage of the approach for software code generation.
It can be skipped depending the results of \ref{main_usage}.

Which criteria for software design and implementation are covered by the methodology
(see EN50128 table A.4) :

\begin{tabular}{|l | c | c | c | c|}
\hline
& \textcolor{green}{Author} & \textcolor{blue}{Assessor 1} & \textcolor{magenta}{Assessor 2} & Total \\
\hline
Formal methods  & & & &  \\
\hline 
Modeling  & & & &  \\
\hline
Modular approach (mandatory) & & & &  \\
\hline
Components & & & &  \\
\hline
Design and coding standards (mandatory) & & & &  \\
\hline
Strongly typed programming language & & & &  \\
\hline

\end{tabular}



\section{Main usage of the tool}
\label{main_usage}

This section discusses the main usage of the tool.

Which task are covered by the tool ?


\begin{tabular}{|l | c | c | c | c|}
\hline
& \textcolor{green}{Author} & \textcolor{blue}{Assessor 1} & \textcolor{magenta}{Assessor 2} & Total \\
\hline 
Modelling support & & & &  \\
\hline
Automatic translation  & & & & \\
\hline
Code Generation  & & & & \\
\hline
Model verification & & & & \\
\hline
Test generation & & & & \\
\hline
Simulation, execution, debugging & & & & \\
\hline
Formal proof & & & & \\
\hline
\end{tabular}

\paragraph{Modelling support}
Does the tool provide a  textual or a graphical editor ?

\paragraph{Automatic translation and code generation}
Which translation or code generation is supported by the tool ?

\paragraph{Model verification}
Which verification on models are provided by the tool?

\paragraph{Test generation}
Does the tool allow to generate tests ? For  which purpose ?

\paragraph{Simulation, execution, debugging}
Does the tool allow to simulate or to debbug step by step a model or a code ?

\paragraph{Formal proof}
Does the tool allow formal proof ?  How ?



\section{Use of the tool}


According WP2 requirements, give a note for characteristics of the use of the tool (from 0 to 3) :

\begin{tabular}{|l | c | c | c | c|}
\hline
& \textcolor{green}{Author} & \textcolor{blue}{Assessor 1} & \textcolor{magenta}{Assessor 2} & Total \\
\hline 
Open Source (D2.6-X-36) & & & &  \\
\hline 
Portability to operating systems (D2.6-X-37) & & & &  \\
\hline
Cooperation of tools (D2.6-X-38) & & & &  \\
\hline
Robustness (D2.6-X-41) & & & & \\
\hline
Modularity (D2.6-X-41.1) & & & & \\
\hline
Documentation management (D.2.6-X-41.2) & & & & \\
\hline
Distributed software development (D.2.6-X-41.3)  & & & & \\
\hline
Simultaneous multi-users (D.2.6-X-41.4)   & & & & \\
\hline
Issue tracking (D.2.6-X-41.5) & & & & \\
\hline
Differences between models (D.2.6-X-41.6) & & & & \\
\hline
Version management (D.2.6-X-41.7) & & & & \\
\hline
Concurrent version development (D.2.6-X-41.8) & & & & \\
\hline
Model-based version control (D.2.6-X-41.9) & & & & \\
\hline
Role traceability (D.2.6-X-41.10) & & & & \\
\hline
Safety version traceability (D.2.6-X-41.11) & & & & \\
\hline
Model traceability (D.2.6-01-035) & & & & \\
\hline
Tool chain integration & & & & \\
\hline
Scalability & & & & \\
\hline
\end{tabular}

\section{Certifiability}

This section discusses how the tool can be classified according EN50128 requirements (D.2.6-X-50).


\begin{tabular}{|l | c | c | c | c|}
\hline
& \textcolor{green}{Author} & \textcolor{blue}{Assessor 1} & \textcolor{magenta}{Assessor 2} & Total \\
\hline 
Tool manual (D.2.6-01-42.02) & & & &  \\
\hline
Proof of correctness (D.2.6-01-42.03)   & & & & \\
\hline
Existing industrial  usage  & & & & \\
\hline
Model verification & & & & \\
\hline
Test generation & & & & \\
\hline
Simulation, execution, debugging & & & & \\
\hline
Formal proof & & & & \\
\hline
\end{tabular}

\paragraph{Other elements for tool certification}

\section{Other comments}
Please to  give free comments on the approach.



