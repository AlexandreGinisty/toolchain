\chapter{GOPRR}

\begin{description}
\item[\textcolor{green}{Author}] Author of the approaches description
  Johannes Feuser/C\'ecile Braunstein  (Uni. Bremen)
\item[\textcolor{blue}{Assessor 1}] First assessor of the approaches Alexandre Ginisty (All4Tec)
\item[\textcolor{magenta}{Assessor 2}] Second assessor of the approaches Matthias G\"udemann (Systerel)
\end{description}

In the sequel, main text is under the responsibilities of the author.

\begin{author_comment}
Author can add comments using this format at any place.
\end{author_comment}

\begin{assessor1}
First assessor can add comments using this format at any place.
\end{assessor1}

\begin{assessor2}
Second assessor can add comments using this format at any place.
\end{assessor2}

When a note is required, please follow this list :
\begin{description}
\item[0] not recommended, not adapted, rejected
\item[1] weakly recommended, adapted after major improvements, weakly rejected
\item[2] recommended, adapted (with light improvements if necessary)  weakly accepted
\item[3] highly recommended, well adapted,strongly accepted
\item[*] difficult to evaluate with a note (please add a comment under the table)
\end{description}

All the notes can be commented under each table.

\section{Presentation}

This section gives a quick presentation of the approach and the tool.

\begin{description}
\item[Name] An openETCS donain specific language Based on GOPPRR
\item[Web site] http://www.informatik.uni-bremen.de/agbs/jfeuser/
\item[Licence] GPL V3
\end{description}

\paragraph{Abstract} 
The approach proposes an open  domain specific language for
openETCS. The language is based on the meta-meta-model  GOPPRR, and the
tool used is MetaEdit+.


\paragraph{Publications} Short list of publications on the approach (5 max)
\begin{itemize}
\item ``MetaEdit+ Workbench User’s Guide'' accessed April 27, 2011. [Online].:
http://www.metacase.com/support/45/manuals/mwb/Mw.html
\item S. Kelly and J.-P. Tolvanen, ``Domain-Specific Modeling''. JOHN WILEY \& SONS, INC.,
2008.
\item J. Feuser and J. Peleska, ``Model Based Development and Tests for openETCS Applications
– A Comprehensive Tool Chain'', 12 2012, in in Proceedings of FORMS/FORMAT 2012.
\item J. Feuser, ``Open source Software for Train Control and
  Application and its Architectural Implication'', PhD. Thesis, 2013.
\end{itemize}


\section{Main usage of the approach}
\label{main_usage}
This section discusses the main usage of the approach.

According to the figure \ref{fig:main_process}, for which phases do you recommend the approach (give a note from 0 to  3) :

\begin{tabular}{|l | c | c | c | c|}
\hline
& \textcolor{green}{Author} & \textcolor{blue}{Assessor 1} & \textcolor{magenta}{Assessor 2} & Total \\
\hline 
System Analysis &
2 &2 & 1&  \\
\hline
Sub-system formal design &3 &3 & 3& \\
\hline
Software design &3 &3 & 3& \\
\hline
Software code generation &3 &3 & 3& \\
\hline
\end{tabular}

According to the figure \ref{fig:main_process}, for which type of activities do you recommend the approach (give a note from 0 to  3) :

\begin{tabular}{|l | c | c | c | c|}
\hline
& \textcolor{green}{Author} & \textcolor{blue}{Assessor 1} & \textcolor{magenta}{Assessor 2} & Total \\
\hline 
Documentation &1 &1 & 1 &  \\
\hline
Modeling &3 &3 & 3&  \\
\hline
Design &2 &2 & 2 & \\
\hline
Code generation &3 &3 & 3& \\
\hline
Verification &0 &0 & 0& \\
\hline
Validation &0 &0 & 0& \\
\hline
Safety analysis &0 &0 & 0& \\
\hline
\end{tabular}

\paragraph{Known usages} Have you some examples of usage of this approach to  compare with the OpenETCS objectives ?

\section{Language}
This section discusses the main element of the language.

Which are the main characteristics of the language :

\begin{tabular}{|l | c | c | c | c|}
  \hline
  & \textcolor{green}{Author} & \textcolor{blue}{Assessor 1} & \textcolor{magenta}{Assessor 2} & Total \\
  \hline 
  Informal language &0 &0 & 0&  \\
  \hline 
  Semi-formal language &0 &0 & 0&  \\
  \hline
  Formal language &3 &2 & 3&  \\
  \hline
  Structured language &3 &3 & 3& \\
  \hline
  Modular language &3 &3 & 3& \\
  \hline
  Textual language &0 &0 & 0& \\
  \hline
  Mathematical symbols or code &0 &0 & 0& \\
  \hline
  Graphical language &3 &3 & 3& \\
  \hline
\end{tabular}

According WP2 requirements, give a note for the capabilities of the language (from 0 to 3) :

\begin{tabular}{|l | c | c | c | c|}
  \hline
  & \textcolor{green}{Author} & \textcolor{blue}{Assessor 1} & \textcolor{magenta}{Assessor 2} & Total \\
  \hline
  Declarative formalization of properties (D.2.6-X-28) &3 &2 & 2& \\
  \hline
  Simple formalization of properties (D.2.6-X-28.1) &2 &2 & 2& \\
  \hline
  Scalability : capability to design large model &3 &3 & 2& \\
  \hline
  Easily translatable to other languages (D.2.6-X-30) &3 &3 & 3& \\
  \hline
  Executable directly (D.2.6-X-33) &0 &0 & 0& \\
  \hline
  Executable after translation to a code (D.2.6-X-33) &3 &3 & 3& \\
  (precise if the translation is automatic) &3 &2 & 3& \\
  \hline
  Simulation, animation (D.2.6-X-33) &0 &0 & 0& \\
  \hline
  Easily understandable (D.2.6-X-27) &2 &2 & 2& \\
  \hline
  Expertise level needed (0 High level, 3 few level) &2 &2 & 2& \\
  \hline
  Standardization (D.2.6-X-29) &* &* & *& \\
  \hline
  Documented (D.2.6-X-29) &3 &2 & 2& \\
  \hline
  Extensible language (D.2.6-01-28) &3 &3 & 2*& \\
  \hline
\end{tabular}

\textcolor{magenta}{Assessor 2} wrt. extensibility: the language is designed to
be specific for ETCS modeling.

\begin{author_comment}
(*) The meta-meta model (GOPPRR) is not a standard but it is formally defined.
\end{author_comment}

\paragraph{Documentation} Describe how the language is documented, the existing guidelines, coding rules, standardization...

The language is fully documented, and the meta-model of the language
is given.
\paragraph{Language usage} Describe the possible restriction on the language

\section{System Analysis}
This section discusses the usage of the approach for system analysis.
It can be skipped depending the results of \ref{main_usage}.

Acoording WP2 requirements, how the approach can be involved for the sub-system requirement specification ?

\begin{tabular}{|l | c | c | c | c|}
\hline
& \textcolor{green}{Author} & \textcolor{blue}{Assessor 1} & \textcolor{magenta}{Assessor 2} & Total \\
\hline
Independent System functions definition (D.2.6-X-10.2.1)  &2 &2 & 2&  \\
\hline 
System architecture design (D.2.6-X-10.2) &3 &3 & 3&  \\
\hline
System data flow identification (D.2.6-X-10.2.3)  &3 &3 & 3&  \\
\hline
Sub-system focus (D.2.6-X-10.2.4)  &3 &3 & 3&  \\
\hline
System interfaces definition (D.2.6-X-10.2.5)  &3 &3 & 3&  \\
\hline
System requirement allocation (D.2.6-X-10.3)  &3 &3 & 2&  \\
\hline
Traceability with SRS (D.2.6-X-10.5)  &2 &2 & 2&  \\
\hline
Traceability with Safety activities (D.2.6-X-11)  &2 &2 & 2 &  \\
\hline
\end{tabular}



\section{Sub-System formal design}
This section discusses the usage of the approach for sub-system formal design.
It can be skipped depending the results of \ref{main_usage}.

Two kinds of model can be planned during this phase: semi-formal models to  cover the SSRS (D.2.6-X-12.1) and strictly formal  models to  focuss on some functional and safety aspects (D.2.6-X-14).  Obviously some strictly  formal means can be used to define the semi-formal  model.

\subsection{Semi-formal model}

Concerning semi-formal model, how the WP2 requirements are covered ?

\begin{tabular}{|l | c | c | c | c|}
\hline
& \textcolor{green}{Author} & \textcolor{blue}{Assessor 1} & \textcolor{magenta}{Assessor 2} & Total \\
\hline 
Consistency to SSRS (D.2.6-X-12.2) & & & &  \\
\hline
Coverage of SSRS (D.2.6-X-12.2.1)  & & & &  \\
\hline
Coverage of SSHA (D.2.6-X-12.2.2)  & & & &  \\
\hline
Management of requirement justification (D.2.6-X-12.2.3)  & & & &  \\
\hline
Traceability to  SSRS (D.2.6-X-12.2.5)  & & & &  \\
\hline
Traceability of exported requirements (D.2.6-X-12.2.6)  & & & &  \\
\hline
Simulation or animation (D.2.6-X-13 partial)  & & & &  \\
\hline
Execution (D.2.6-X-13 partial)  & & & &  \\
\hline
Extensible to strictly formal model (D.2.6-X-14.3) & & & &  \\
\hline
Easy to  refine towards strictly formal model (D.2.6-X-14.4) & & & &  \\
\hline
Extensible and modular design (D.2.6-X-15)  & & & &  \\
\hline
Extensible to software architecture and design (D.2.6-X-30)   & & & &  \\
\hline
\end{tabular}

Concerning safety properties management, how the WP2 requirements are covered ?

\begin{tabular}{|l | c | c | c | c|}
\hline
& \textcolor{green}{Author} & \textcolor{blue}{Assessor 1} & \textcolor{magenta}{Assessor 2} & Total \\
\hline 
Safety function isolation (D.2.6-X-17)  & & & &  \\
\hline 
Safety properties formalisation (D.2.6-X-22)  & & & &  \\
\hline
Logical expression (D.2.6-X-28.2.2)  & & & &  \\
\hline
Timing constraints (D.2.6-X-28.2.3)  & & & &  \\
\hline
Safety properties validation (D.2.6-X-23.2)  & & & &  \\
\hline
Logical properties assertion (D.2.6-X-34)  & & & &  \\
\hline
Check  of assertions (D.2.6-X-34.1)  & & & &  \\
\hline
\end{tabular}

Does the language allow to  formalize (D.2.6-X-31):

\begin{tabular}{|l | c | c | c | c|}
\hline
& \textcolor{green}{Author} & \textcolor{blue}{Assessor 1} & \textcolor{magenta}{Assessor 2} & Total \\
\hline 
State machines  & & & &  \\
\hline
Time-outs  & & & &  \\
\hline
Truth tables  & & & &  \\
\hline
Arithmetic  & & & &  \\
\hline
Braking curves  & & & &  \\
\hline
Logical statements & & & &  \\
\hline
Message and fields & & & &  \\
\hline
\end{tabular}

\paragraph{Additional comments on semi-formal  model} Do you think your semi-formal  model is sufficient to cover a safe design of the on-board unit until code generation ?
All comments on links to  other models, validation and verification activities are welcomed.

\subsection{Strictly formal model}

Concerning strictly formal model, how the WP2 requirements are covered ?

\begin{tabular}{|l | c | c | c | c|}
\hline
& \textcolor{green}{Author} & \textcolor{blue}{Assessor 1} & \textcolor{magenta}{Assessor 2} & Total \\
\hline 
Consistency to SFM (D.2.6-X-14.2) &* &* & *&  \\
\hline
Coverage of SSRS (D.2.6-X-14.2)  &1 &1 & 1&  \\
\hline
Traceability to  SSRS (D.2.6-X-14.3)  &3 &2 & 2&  \\
\hline
Extensible to software design (D.2.6-X-16)  &1 &1 & 1&  \\
\hline
Safety function isolation (D.2.6-X-17)  &1 &1 & 1&  \\
\hline 
Safety properties formalisation (D.2.6-X-22)  &2 &2 & 1&  \\
\hline
Logical expression (D.2.6-X-28.2.2)  &3 &3 & 3&  \\
\hline
Timing constraints (D.2.6-X-28.2.3)  &0 &0 & 0&  \\
\hline
Safety properties validation (D.2.6-X-23.3)  &3 &2 & 2&  \\
\hline
Logical properties assertion (D.2.6-X-34)  &3 &3 & 3&  \\
\hline
Proof of assertions (D.2.6-X-34.2)  &3 &3 & 2&  \\
\hline
\end{tabular}

\begin{author_comment}
(*) Since the semi-formal model is not yet done, it is hard to say.
\end{author_comment}
Does the language allow to  formalize (D.2.6-X-32):

\begin{tabular}{|l | c | c | c | c|}
\hline
& \textcolor{green}{Author} & \textcolor{blue}{Assessor 1} & \textcolor{magenta}{Assessor 2} & Total \\
\hline 
State machines  &3 &3 & 3 &  \\
\hline
Time-outs  &0 &0 & 0&  \\
\hline
Truth tables  &0 &0 & 0&  \\
\hline
Arithmetic  &3 &3 & 3&  \\
\hline
Braking curves  &3 &3 & 3&  \\
\hline
Logical statements &3 &3 & 3&  \\
\hline
Message and fields &3 &3 & 3&  \\
\hline
\end{tabular}

\paragraph{Additional comments on semi-formal  model} Do you think your strictly formal  model can be directly defined from the SSRS ?
All comments on links to  other models, validation and verification activities are welcomed.


\section{Software design}
This section discusses the usage of the approach for software design.
It can be skipped depending the results of \ref{main_usage}.

\subsection{Functional design}

How the approach allows to  produce a functional software model of the on-board unit ?

\begin{tabular}{|l | c | c | c | c|}
\hline
& \textcolor{green}{Author} & \textcolor{blue}{Assessor 1} & \textcolor{magenta}{Assessor 2} & Total \\
\hline
Derivation from system semi-formal model  &* &* & *&  \\
\hline 
Software architecture description  &3 &3 & 3&  \\
\hline
Software constraints  &3 &3 & 3&  \\
\hline
Traceability  &3 &3 & 3&  \\
\hline
Executable  &3 &2 & 3&  \\
\hline
\end{tabular}
\begin{author_comment}
(*) Since we do not know the format of the semi-formal model  yet, it is hard to say.
\end{author_comment}

\subsection{SSIL4 design}

How the approach allows to  produce in safety a software model ?

\begin{tabular}{|l | c | c | c | c|}
\hline
& \textcolor{green}{Author} & \textcolor{blue}{Assessor 1} & \textcolor{magenta}{Assessor 2} & Total \\
\hline
Derivation from system semi-formal or strictly formal model  &* &* & *&  \\
\hline 
Software architecture description  &3 &3 & 3&  \\
\hline
Software constraints  &3 &3 & 3&  \\
\hline
Traceability  &3 &3 & 3&  \\
\hline
Executable  &3 &2 & 3&  \\
\hline
Conformance to EN50128 § 7.2  &0 &0 & 0&  \\
\hline
Conformance to EN50128 § 7.3  &0 &0 & 0&  \\
\hline
Conformance to EN50128 § 7.4  &0 &0 & 0&  \\
\hline
\end{tabular}
\begin{author_comment}
(*) Since we do not know the format of the semi-formal model  yet, it is hard to say.
\end{author_comment}

Which criteria for software architecture are covered by the methodology
(see EN50128 table A.3) :

\begin{tabular}{|l | c | c | c | c|}
\hline
& \textcolor{green}{Author} & \textcolor{blue}{Assessor 1} & \textcolor{magenta}{Assessor 2} & Total \\
\hline
Defensive programming  &3 &3 & 1*&  \\
\hline 
Fault detection \& diagnostic  &3 &3 & 1&  \\
\hline
Error detecting code  &3 &3 & 1&  \\
\hline
Failure assertion programming &3 &3 & 3&  \\
\hline
Diverse programming &1 &1 & 1&  \\
\hline
Memorising executed cases &0 &0 & 0&  \\
\hline
Software error effect analysis &0 &0 & 0&  \\
\hline
Fully defined interface &3 &3 & 3&  \\
\hline
Modeling  &3 &3 & 3&  \\
\hline
Structured methodology &3 &3 & 3&  \\
\hline
\end{tabular}

\textcolor{magenta}{Assessor 2} I am not aware of explicit means to defensive
programming, nor are they mentioned in Johannes' thesis.

\section{Software code generation}
This section discusses the usage of the approach for software code generation.
It can be skipped depending the results of \ref{main_usage}.

Which criteria for software design and implementation are covered by the methodology
(see EN50128 table A.4) :

\begin{tabular}{|l | c | c | c | c|}
\hline
& \textcolor{green}{Author} & \textcolor{blue}{Assessor 1} & \textcolor{magenta}{Assessor 2} & Total \\
\hline
Formal methods  &3 &3 & &  \\
\hline 
Modeling  &3 &3 & &  \\
\hline
Modular approach (mandatory) &3 &3 & 3&  \\
\hline
Components &3 &3 & 3&  \\
\hline
Design and coding standards (mandatory) &3 &3 & *&  \\
\hline
Strongly typed programming language &3 &3 & 3&  \\
\hline

\end{tabular}

\textcolor{magenta}{Assessor 2} Are there coding standards?

\section{Main usage of the tool}
\label{main_usage}

This section discusses the main usage of the tool.

Which task are covered by the tool ?


\begin{tabular}{|l | c | c | c | c|}
\hline
& \textcolor{green}{Author} & \textcolor{blue}{Assessor 1} & \textcolor{magenta}{Assessor 2} & Total \\
\hline 
Modelling support &3 &3 & 3&  \\
\hline
Automatic translation  &1 &1 & 1& \\
\hline
Code Generation  &3 &3 & 3& \\
\hline
Model verification &2 &2 & 2& \\
\hline
Test generation &* &* & *& \\
\hline
Simulation, execution, debugging &2 &2 & 2& \\
\hline
Formal proof &0 &0 & 0& \\
\hline
\end{tabular}
\begin{author_comment}
(*) The new version of MetaEdit+ provides this feature.
\end{author_comment}

\paragraph{Modeling support}
Does the tool provide a  textual or a graphical editor ?

Graphical
\paragraph{Automatic translation and code generation}
Which translation or code generation is supported by the tool ?

MERL generator
\paragraph{Model verification}
Which verification on models are provided by the tool?

Inspection

\paragraph{Test generation}
Does the tool allow to generate tests ? For  which purpose ?

No
\paragraph{Simulation, execution, debugging}
Does the tool allow to simulate or to debbug step by step a model or a code ?


The new version yes.
\paragraph{Formal proof}
Does the tool allow formal proof ?  How ?

No

\section{Use of the tool}


According WP2 requirements, give a note for characteristics of the use of the tool (from 0 to 3) :

\begin{tabular}{|l | c | c | c | c|}
\hline
& \textcolor{green}{Author} & \textcolor{blue}{Assessor 1} & \textcolor{magenta}{Assessor 2} & Total \\
\hline 
Open Source (D2.6-X-36) &0 &0 & 0&  \\
\hline 
Portability to operating systems (D2.6-X-37) &3 &3 & 3&  \\
\hline
Cooperation of tools (D2.6-X-38) &1 &1 & 1&  \\
\hline
Robustness (D2.6-X-41) &3 &3 & *& \\
\hline
Modularity (D2.6-X-41.1) &0 &0 & 0& \\
\hline
Documentation management (D.2.6-X-41.2) &0 &0 & 0& \\
\hline
Distributed software development (D.2.6-X-41.3)  &3 &3 & 3& \\
\hline
Simultaneous multi-users (D.2.6-X-41.4)   &3 &3 & 3& \\
\hline
Issue tracking (D.2.6-X-41.5) &0 &0 & 0& \\
\hline
Differences between models (D.2.6-X-41.6) &0 &0 & 0& \\
\hline
Version management (D.2.6-X-41.7) &1 &1 & 1& \\
\hline
Concurrent version development (D.2.6-X-41.8) &0 &0 & 0& \\
\hline
Model-based version control (D.2.6-X-41.9) &0 &0 & 0& \\
\hline
Role traceability (D.2.6-X-41.10) &0 &0 & 0& \\
\hline
Safety version traceability (D.2.6-X-41.11) &0 &0 & 0& \\
\hline
Model traceability (D.2.6-01-035) &1 &1 & 1& \\
\hline
Tool chain integration &3 &3 & *& \\
\hline
Scalability &3 &3 & & \\
\hline
\end{tabular}

\textcolor{magenta}{Assessor 2} Could not test the robustness of Metaedit+.

Seems to be a stand-alone tool, so toolchain integration is difficult to assess
without trying the tool.

\section{Certifiability}

This section discusses how the tool can be classified according EN50128 requirements (D.2.6-X-50).


\begin{tabular}{|l | c | c | c | c|}
\hline
& \textcolor{green}{Author} & \textcolor{blue}{Assessor 1} & \textcolor{magenta}{Assessor 2} & Total \\
\hline 
Tool manual (D.2.6-01-42.02) &3 &3 & 3&  \\
\hline
Proof of correctness (D.2.6-01-42.03)   &0 &0 & 0& \\
\hline
Existing industrial  usage  &3 &3 & 3& \\
\hline
Model verification &1 &1 & 1& \\
\hline
Test generation &0 &0 & 0& \\
\hline
Simulation, execution, debugging &0 &0 & 0& \\
\hline
Formal proof &0 &0 & 0& \\
\hline
\end{tabular}

\paragraph{Other elements for tool certification}

\section{Other comments}
Please to  give free comments on the approach.




%  LocalWords:  MetaEdit GOPPRR
