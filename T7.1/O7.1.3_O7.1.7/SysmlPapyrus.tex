\chapter{SysML with Papyrus}

\begin{description}
\item[\textcolor{green}{Author}] Author of the approaches description : CEA Nano-Innov labs (Agnes Lanusse, Mathieu Perrin and Armand Nachef) and All4tec (Alexandre Ginisty and Cyril Cornu)
\item[\textcolor{blue}{Assessor 1}] Renaud De Landsheer (Alstom BE)
\item[\textcolor{magenta}{Assessor 2}] Marielle Petit-Doche (Systerel)
\end{description}

In the sequel, main text is under the responsibilities of the author.

\begin{author_comment}
Author can add comments using this format
\end{author_comment}

\begin{assessor1}
First assessor can add comments using this format
\end{assessor1}

\begin{assessor2}
Second assessor can add comments using this format
\end{assessor2}

When a note is required, please follow this list :
\begin{description}
\item[0] not recommended, not adapted, rejected
\item[1] weakly recommended, adapted after major improvements, weakly rejected
\item[2] recommended, adapted (with light improvements if necessary) weakly accepted
\item[3] highly recommended, well adapted,strongly accepted
\item[*] difficult to evaluate with a note (please add a comment under the table)
\end{description}

All the notes can be commented under each table.

\section{Presentation}

This section gives a quick presentation of the approach.

\begin{description}
\item[Name] Name of the approach
\item[Web site] \url{http://www.eclipse.org/papyrus/}
\item[Licence] Open source : EUPL
\end{description}

\paragraph{Abstract} Short abstract on the approach and tool (10 lines max)

\paragraph{Publications} Short list of publications on the approach (5 max)


\section{Main usage of the approach}
\label{main_usage}
This section discusses the main usage of the approach.

According to the figure \ref{fig:main_process}, for which phases do you recommend the approach (give a note from 0 to 3) :

\begin{tabular}{|l | c | c | c | c|}
\hline
& \textcolor{green}{Author} & \textcolor{blue}{Assessor 1} & \textcolor{magenta}{Assessor 2} & Total \\
\hline
System Analysis & 2 & 3 & 2 & 7 \\
\hline
Sub-system formal design & 3 & 1 & 2 & 6 \\
\hline
Software design & 2 & 2 & 2 & 6 \\
\hline
Software code generation & 1 & 1 & 1 & 3 \\
\hline
\end{tabular}

\begin{assessor1}
Concerning the "`Sub-system formal design"', I consider that Papyrus is more a semi-formal too than a fully formal one, so I've put a mark "`2"' for this criterion. 
As a matter of fact, it is quoted "`1"' for use as a formal language below. 
\end{assessor1}

\begin{assessor2}
I agree comment of assesor 1, SysML is a "semi-formal" approach and does not allow design of sub-system "fully formal". Moreover expression of properties is poor in strict SysMl language.
However SysML is good to produce a structured model of the system  and to help  in its analyses, but it shall be completed to cover all the needs of system analysis.
\end{assessor2}

According to the figure \ref{fig:main_process}, for which type of activities do you recommend the approach (give a note from 0 to 3) :

\begin{tabular}{|l | c | c | c | c|}
\hline
& \textcolor{green}{Author} & \textcolor{blue}{Assessor 1} & \textcolor{magenta}{Assessor 2} & Total \\
\hline
Documentation & 1 & 3 & 2 & 6 \\
\hline
Modeling & 3 &  3 & 3 & 9 \\
\hline
Design & 2 &  2 & 2 & 6 \\
\hline
Code generation & 1 &  1 & 1 & 3 \\
\hline
Verification & 3 &  2 & 1 & 6 \\
\hline
Validation & 2 &  2 & 1 & 5 \\
\hline
Safety analyses & 3 & * & 1 & 4* \\
\hline
\end{tabular}

\begin{assessor1}
I've put a rather higher mark on Documentation because the models supported by Papyrus can be used as documentation items. 

Concerning the safety analyses, I do not see how Papyrus helps in driving such analyses. 
\end{assessor1}

\begin{assessor2}
SysML can provide useful elements for documentation, however, a clear methodology shall be provide to understand and drive the models. For verification, validation and safety analysis, SysML alone is not sufficient for critical systems. 
\end{assessor2}

\paragraph{Known usages} Have you some examples of usage of this approach to compare with the OpenETCS objectives ?

\begin{assessor2}
Elements missing
\end{assessor2}


\section{Language}
This section discusses the main element of the language.

According WP2 requirements, give a note for the characteristics of the language (from 0 to 3) :

\begin{tabular}{|l | c | c | c | c|}
\hline
& \textcolor{green}{Author} & \textcolor{blue}{Assessor 1} & \textcolor{magenta}{Assessor 2} & Total \\
\hline
Informal language & 3 & 3 & 3 & 9 \\
\hline
Semi-formal language & 3 & 3 & 3 & 9 \\
\hline
Formal language & 1 & 1 & 1 & 3 \\
\hline
Structured language & 3 & 3 & 3 & 9 \\
\hline
Modular language & 2 & 2 & 2 & 6 \\
\hline
Textual language & 3 & 2 & 2 & 7 \\
\hline
Mathematical symbols or code & 0 & 0 & 0  & 0 \\
\hline
Graphical language & 3 & 3 & 3 & 9 \\
\hline
\end{tabular}

According WP2 requirements, give a note for the capabilities of the language (from 0 to 3) :

\begin{tabular}{|l | c | c | c | c|}
\hline
& \textcolor{green}{Author} & \textcolor{blue}{Assessor 1} & \textcolor{magenta}{Assessor 2} & Total \\
\hline
Declarative formalization of properties (D.2.6-X-28) & 3 (0) & 1 & 1 & 5(2) \\
\hline
Simple formalization of properties (D.2.6-X-28.1) & 3 &  2 & 1 & 6 \\
\hline
Scalability : capability to design large model & 3 & * & 2 & 5* \\
\hline
Easily translatable to other languages (D.2.6-X-30) & 2 & 3 & 2 & 7 \\
\hline
Executable directly (D.2.6-X-33) & 0 & 0 & 0 & 0 \\
\hline
Executable after translation to a code (D.2.6-X-33) & 1 & 2 & * & 3* \\
(precise if the translation is automatic) & & & & \\
\hline
Simulation, animation (D.2.6-X-33) & 2 & 2 & * & 4* \\
\hline
Easily understandable (D.2.6-X-27) & 1 & 2 &  1 & 4 \\
\hline
Expertise level needed (0 High level, 3 few level) & 1 & 2 &  1 & 4 \\
\hline
Standardization (D.2.6-X-29) & 3 & 3 & 3 & 9 \\
\hline
Documented (D.2.6-X-29) & 2 & 2 & 2 & 6 \\
\hline
Extensible language (D.2.6-01-28) & 3 & 3 & 3 & 9 \\
\hline
\end{tabular}

\begin{assessor1}
In my understanding, if a language supports the declarative formalization of properties there should be the possibility to write some mathematical symbols, or code, so I do not understand the notations provided by the authors in the points
"`Declarative formalization of properties"', and "`Mathematical symbols or code"'. I see that declarative properties could be attached to all state, as pre or post-condition, or as "`constraints"' to any ActivityFigure. The language encompass some timing properties, and seemignly arbitrary expressions, but again, the language seems partially formal. Also, the editing actions necessary to add such constraint is rather complex. 
\end{assessor1}

\begin{assessor2}
Expression of properties is difficult in sysML, and the OCL language is very poor. Besides there are no good tool to check the properties. 

I have no see example of execution, simulation,...

Concerning scalability, translation, understanding or expertise level, SysML can not be used with a good methodology guide (standard are not sufficient)  to  precisely define how and with which elements we can draw models. If a model as been defined without following a set of predefined rules, it can not be easily translate in an another language, or only  partly.
\end{assessor2}

\paragraph{Documentation} Describe how the language is documented, the existing guidelines, coding rules, standardization...


\begin{assessor2}
Elements missing
\end{assessor2}


\paragraph{Language usage} Describe the possible restriction on the language


\begin{assessor2}
Elements missing
\end{assessor2}



\section{System Analysis}
This section discusses the usage of the approach for system analysis.
It can be skipped depending the results of \ref{main_usage}.

Acoording WP2 requirements, how the approach can be involved for the sub-system requirement specification ?

\begin{tabular}{|l | c | c | c | c|}
\hline
& \textcolor{green}{Author} & \textcolor{blue}{Assessor 1} & \textcolor{magenta}{Assessor 2} & Total \\
\hline
Independent System functions definition (D.2.6-X-10.1.1) & 1 & 1 & 2 & 4 \\
\hline
System architecture design (D.2.6-X-10.1.2) & 3 & 3 & 3  & 9 \\
\hline
System data flow identification (D.2.6-X-10.1.3) & 2 & 2 & 2 & 6 \\
\hline
Sub-system focus (D.2.6-X-10.1.4) & 3 & 3 & 3 & 9 \\
\hline
System interfaces definition (D.2.6-X-10.1.5) & 3 & 3 & 3 &  9\\
\hline
System requirement allocation (D.2.6-X-10.2) & 3 & 3 & 3 & 9 \\
\hline
Traceability with SRS (D.2.6-X-10.3) & 3 & 2 & 1 & 6 \\
\hline
Traceability with Safety activities (D.2.6-X-11) & 3 & 2 & 1 & 6 \\
\hline
\end{tabular}

\begin{assessor1}
concerning the "`Traceability with SRS"', this requires that the SSRS is encoded in the formalism of Papyrus. I just wonder how the Subset26 can be encoded with its structure, into the requirements formalism of Papyrus. 
\end{assessor1}


\begin{assessor2}
In the given examples it is very difficult to check the traceability with the SRS or SSRS (lots of comments are missing). Besides, how can we check the coverage of SRS requirement by the SysML model ?
\end{assessor2}



\section{Sub-System formal design}
This section discusses the usage of the approach for sub-system formal design.
It can be skipped depending the results of \ref{main_usage}.

Two kinds of model can be planned during this phase: semi-formal models to cover the SSRS (D.2.6-X-12.1) and strictly formal models to focuss on some functional and safety aspects (D.2.6-X-14). Obviously some strictly formal means can be used to define the semi-formal model.

\subsection{Semi-formal model}

Concerning semi-formal model, how the WP2 requirements are covered ?

\begin{tabular}{|l | c | c | c | c|}
\hline
& \textcolor{green}{Author} & \textcolor{blue}{Assessor 1} & \textcolor{magenta}{Assessor 2} & Total \\
\hline
Consistency to SSRS (D.2.6-X-12.2) & 2 & 2 & 2 & 6 \\
\hline
Coverage of SSRS (D.2.6-X-12.2.1) & 2 & 2 & 1 & 5 \\
\hline
Coverage of SSHA (D.2.6-X-12.2.2) & 2 & 2 & 1 & 5 \\
\hline
Management of requirement justification (D.2.6-X-12.2.3) & 3 & 3 & 3 & 9 \\
\hline
Traceability to SSRS (D.2.6-X-12.2.5) & 3 & 3 & 2 & 8 \\
\hline
Traceability of exported requirements (D.2.6-X-12.2.6) & 3 & 3 & 2 & 8 \\
\hline
Simulation or animation (D.2.6-X-13 partial) & 2 & 2 &  * & 4* \\
\hline
Execution (D.2.6-X-13 partial) & 0 & 0 & * & 0* \\
\hline
Extensible to strictly formal model (D.2.6-X-14.3) & 1 & 1 & 1 & 3 \\
\hline
Easy to refine towards strictly formal model (D.2.6-X-14.4) & 1 & 1 & 2 & 4 \\
\hline
Extensible and modular design (D.2.6-X-15) & 3 & 3 & 3 & 9 \\
\hline
Extensible to software architecture and design (D.2.6-X-30) & 3 & 3 & 2 & 8 \\
\hline
\end{tabular}



\begin{assessor2}
Same comments as below on execution and simulation.

Current state of SysML language do not allow a good management of the traceability, expecially coverage topics.

It is difficult to extend SysML to strictly formal model, however it can be refined to a formal model on some parts if some methodological constraints have been followed.
\end{assessor2}

Concerning safety properties management, how the WP2 requirements are covered ?

\begin{tabular}{|l | c | c | c | c|}
\hline
& \textcolor{green}{Author} & \textcolor{blue}{Assessor 1} & \textcolor{magenta}{Assessor 2} & Total \\
\hline
Safety function isolation (D.2.6-X-17) & 3 & 3 & 3 & 9 \\
\hline
Safety properties formalisation (D.2.6-X-22) & 3 & * & 1 & 4* \\
\hline
Logical expression (D.2.6-X-28.2.2) & 3 & 1 & 1 & 5 \\
\hline
Timing constraints (D.2.6-X-28.2.3) & 1 & 1 & 1 & 3 \\
\hline
Safety properties validation (D.2.6-X-23.2) & 0 & 0 & 0 & 0 \\
\hline
Logical properties assertion (D.2.6-X-34) & 3 & 0 & 0 & 3 \\
\hline
Check of assertions (D.2.6-X-34.1) & 0 & 0 &  0 & 0 \\
\hline
\end{tabular}




\begin{assessor2}
SysML alone is to  poor to deal with safety aspects.
\end{assessor2}


Does the language allow to formalize (D.2.6-X-31):

\begin{tabular}{|l | c | c | c | c|}
\hline
& \textcolor{green}{Author} & \textcolor{blue}{Assessor 1} & \textcolor{magenta}{Assessor 2} & Total \\
\hline
State machines & 2 & 3 & 3 & 8 \\
\hline
Time-outs & 2 & 2 & 2 & 6 \\
\hline
Truth tables & 1 & 1 & 1 & 3 \\
\hline
Arithmetic & 0 & 0 & 0 & 0 \\
\hline
Braking curves & 0 & 0 & 0 & 0 \\
\hline
Logical statements & 3 & 3 & 1 & 7 \\
\hline
Message and fields & 3 & 3 & 3 & 9 \\
\hline
\end{tabular}

\begin{assessor1}
I feel a bit uncomfortable saying what can be formalized in a semi-formal way, so I'v interpreted this as "`represented"' in a semi-formal way. 
\end{assessor1}

\paragraph{Additional comments on semi-formal model} Do you think your semi-formal model is sufficient to cover a safe design of the on-board unit until code generation ?
All comments on links to other models, validation and verification activities are welcomed.


\begin{assessor2}
I think SysML is good to give a clear structure of the system and its architecture, but insufficient to cover all the process and safety aspects.
\end{assessor2}

\subsection{Strictly formal model}

Concerning strictly formal model, how the WP2 requirements are covered ?

\begin{assessor1}
SysML/Papyrus is obviously not a strictly formal tool, I therefor skip this section. 
This is illustrated by the evaluation of the author in the second table of this section, stating that nothing can be formalized in SysML/Papyrus. 
\end{assessor1}


\begin{assessor2}
I agree assessor 1, section skip too.
\end{assessor2}

\begin{tabular}{|l | c | c | c | c|}
\hline
& \textcolor{green}{Author} & \textcolor{blue}{Assessor 1} & \textcolor{magenta}{Assessor 2} & Total \\
\hline
Consistency to SFM (D.2.6-X-14.2) & 1 & & & - \\
\hline
Coverage of SSRS (D.2.6-X-14.2) & 2 & & & - \\
\hline
Traceability to SSRS (D.2.6-X-14.3) & 2 && & - \\
\hline
Extensible to software design (D.2.6-X-16) & 3 & & & - \\
\hline
Safety function isolation (D.2.6-X-17) & 2 & & & - \\
\hline
Safety properties formalisation (D.2.6-X-22) & 1 & & & - \\
\hline
Logical expression (D.2.6-X-28.2.2) & 3 & & & - \\
\hline
Timing constraints (D.2.6-X-28.2.3) & 1 & & & - \\
\hline
Safety properties validation (D.2.6-X-23.3) & 1 & & & - \\
\hline
Logical properties assertion (D.2.6-X-34) & 3 & & & - \\
\hline
Proof of assertions (D.2.6-X-34.2) & 0 & & & - \\
\hline
\end{tabular}

Does the language allow to formalize (D.2.6-X-32):

\begin{tabular}{|l | c | c | c | c|}
\hline
& \textcolor{green}{Author} & \textcolor{blue}{Assessor 1} & \textcolor{magenta}{Assessor 2} & Total \\
\hline
State machines & 1 & & & - \\
\hline
Time-outs & 0 &  & & - \\
\hline
Truth tables & 0 & & & - \\
\hline
Arithmetic & 0 & & & - \\
\hline
Braking curves & 0 & & & - \\
\hline
Logical statements & 0 & & & - \\
\hline
Message and fields & 0 & & & - \\
\hline
\end{tabular}

\paragraph{Additional comments on semi-formal model} Do you think your strictly formal model can be directly defined from the SSRS ?
All comments on links to other models, validation and verification activities are welcomed.

\section{Software design}
This section discusses the usage of the approach for software design.
It can be skipped depending the results of \ref{main_usage}.

\subsection{Functional design}

How the approach allows to produce a functional software model of the on-board unit ?

\begin{tabular}{|l | c | c | c | c|}
\hline
& \textcolor{green}{Author} & \textcolor{blue}{Assessor 1} & \textcolor{magenta}{Assessor 2} & Total \\
\hline
Derivation from system semi-formal model & 3 & 3 & 3 & 9 \\
\hline
Software architecture description & 3 & 3 & 2 & 8 \\
\hline
Software constraints & 3 & 2 & 1 & 6 \\
\hline
Traceability & 2 & 2 & 2 & 6 \\
\hline
Executable & 1 & 1 & * & 2* \\
\hline
\end{tabular}



\begin{assessor2}
Same comments as bellow on executable.

SysMl does not seems adapted to take into account software constraints to define a performing code.
\end{assessor2}

\subsection{SSIL4 design}

How the approach allows to produce in safety a software model ?

\begin{tabular}{|l | c | c | c | c|}
\hline
& \textcolor{green}{Author} & \textcolor{blue}{Assessor 1} & \textcolor{magenta}{Assessor 2} & Total \\
\hline
Derivation from system semi-formal or strictly formal model & 1 & 1 & 1 & 3 \\
\hline
Software architecture description & 3 & 3 & 3 & 9 \\
\hline
Software constraints & 2 & 1 & 1 & 4 \\
\hline
Traceability & 1 & 2 & 2 & 5 \\
\hline
Executable & 0 & 0 & * & 0* \\
\hline
Conformance to EN50128 § 7.2 & 3 & * & 2 & 5* \\
\hline
Conformance to EN50128 § 7.3 & 3 & * & 1 & 4* \\
\hline
Conformance to EN50128 § 7.4 & 3 & * & 1 & 4* \\
\hline
\end{tabular}

\begin{assessor1}
I have strictly no experience with norms, so I do not feel able to put a proper evaluation for EN50128 compliance. Since the code is not executable, I do not see the point of putting a criterion for these items anyway. 
\end{assessor1}


\begin{assessor2}
According results of the criteria on table A.3 and A.4, SysML alone is not sufficient for a critical software design activity.
\end{assessor2}


Which criteria for software architecture are covered by the methodology
(see EN50128 table A.3) :

\begin{tabular}{|l | c | c | c | c|}
\hline
& \textcolor{green}{Author} & \textcolor{blue}{Assessor 1} & \textcolor{magenta}{Assessor 2} & Total \\
\hline
Defensive programming & 0 & 0 & 0 & 0 \\
\hline
Fault detection \& diagnostic & 0 & 0 & 0 & 0 \\
\hline
Error detecting code & 0 & 0 & 0 & 0 \\
\hline
Failure assertion programming & 2 & 2 & 2 & 6 \\
\hline
Diverse programming & 0 & 0 & 0 & 0 \\
\hline
Memorising executed cases & 0 & 0 & 0 & 0 \\
\hline
Software error effect analysis & 2 & 0 & 0 & 2 \\
\hline
Fully defined interface & 3 & 3 & 3 & 9 \\
\hline
Modelling & 3 & 3 & 2 & 8 \\
\hline
Structured methodology & 3 & 2 & 3 & 8 \\
\hline
\end{tabular}

\section{Software code generation}
This section discusses the usage of the approach for software code generation.
It can be skipped depending the results of \ref{main_usage}.

Which criteria for software design and implementation are covered by the methodology
(see EN50128 table A.4) :

\begin{tabular}{|l | c | c | c | c|}
\hline
& \textcolor{green}{Author} & \textcolor{blue}{Assessor 1} & \textcolor{magenta}{Assessor 2} & Total \\
\hline
Formal methods & 0 & 0 & 0 & 0 \\
\hline
Modeling & 1 & 2 & 1 & 4 \\
\hline
Modular approach (mandatory) & 2 & 2 & 2 & 6 \\
\hline
Components & 1 & 1 & 1 & 3 \\
\hline
Design and coding standards (mandatory) & 0 & 0 & 1 & 1 \\
\hline
Strongly typed programming language & 3 & 3 & 3 & 9 \\
\hline
\end{tabular}

\section{Main usage of the tool}
\label{main_usage}

This section discusses the main usage of the tool.

Which task are covered by the tool ?


\begin{tabular}{|l | c | c | c | c|}
\hline
& \textcolor{green}{Author} & \textcolor{blue}{Assessor 1} & \textcolor{magenta}{Assessor 2} & Total \\
\hline
Modelling support & 3 & 3 & 3 & 9 \\
\hline
Automatic translation & 1 & 1 & 0 & 2 \\
\hline
Code Generation & 0 & 0 & 0 & 0 \\
\hline
Model verification & 0 & 0 & 0 & 0 \\
\hline
Test generation & 2 & 2 & 1 & 5 \\
\hline
Simulation, execution, debugging & 0 & 0 & 0 & 0 \\
\hline
Formal proof & 0 & 0 & 0 & 0 \\
\hline
\end{tabular}

\paragraph{Modelling support}
Papyrus provides a graphical editor.

\paragraph{Automatic translation and code generation}
Which translation or code generation is supported by the tool ?

\paragraph{Model verification}
Which verification on models are provided by the tool?

\paragraph{Test generation}
Does the tool allow to generate tests ? For which purpose ?

\paragraph{Simulation, execution, debugging}
Does the tool allow to simulate or to debbug step by step a model or a code ?

\paragraph{Formal proof}
Does the tool allow formal proof ? How ?

\section{Use of the tool}

According WP2 requirements, give a note for characteristics of the use of the tool (from 0 to 3) :

\begin{tabular}{|l | c | c | c | c|}
\hline
& \textcolor{green}{Author} & \textcolor{blue}{Assessor 1} & \textcolor{magenta}{Assessor 2} & Total \\
\hline
Open Source (D2.6-X-36) & 3 & 3 & 3 & 9 \\
\hline
Portability to operating systems (D2.6-X-37) & 1 & 2 &  3 & 6 \\
\hline
Cooperation of tools (D2.6-X-38) & 3 & 2 & 3 & 8 \\
\hline
Robustness (D2.6-X-41) & 1 & 1 & 1 & 3 \\
\hline
Modularity (D2.6-X-41.1) & 3 & 3 & 2 & 8 \\
\hline
Documentation management (D.2.6-X-41.2) & 2 & 2 & 1  & 5 \\
\hline
Distributed software development (D.2.6-X-41.3) & 2 & 2 & 2 & 6 \\
\hline
Simultaneous multi-users (D.2.6-X-41.4) & 2 & 2 & 2 & 6 \\
\hline
Issue tracking (D.2.6-X-41.5) & 1 & 2 & 1 & 4 \\
\hline
Differences between models (D.2.6-X-41.6) & 1 & 2 & 1 & 4 \\
\hline
Version management (D.2.6-X-41.7) & 2 & 2 & 1 & 5 \\
\hline
Concurrent version development (D.2.6-X-41.8) & 2 & 2 & 2 &  6\\
\hline
Model-based version control (D.2.6-X-41.9) & 1 & 1 & 1 & 3 \\
\hline
Role traceability (D.2.6-X-41.10) & 2 & 1 & 1 & 4 \\
\hline
Safety version traceability (D.2.6-X-41.11) & 2 & 2 & 1 & 5 \\
\hline
Model traceability (D.2.6-01-035) & 3 & 2 & 2 & 7 \\
\hline
Tool chain integration & 3 & 3 & 3 & 9 \\
\hline
Scalability & 3 & 2 & 1 & 6 \\
\hline
\end{tabular}

\section{Certifiability}

This section discusses how the tool can be classified according EN50128 requirements (D.2.6-X-50).


\begin{tabular}{|l | c | c | c | c|}
\hline
& \textcolor{green}{Author} & \textcolor{blue}{Assessor 1} & \textcolor{magenta}{Assessor 2} & Total \\
\hline
Tool manual (D.2.6-01-42.02) & 3 & 2 & 2 & 7 \\
\hline
Proof of correctness (D.2.6-01-42.03) & 1 & 0 & 0  & 1 \\
\hline
Existing industrial usage & 2 & 2 & 1 & 5 \\
\hline
Model verification & 0 & 0 & 0 & 0 \\
\hline
Test generation & 2 & 2 & 1 & 5 \\
\hline
Simulation, execution, debugging & 2 & 2 & 1 & 5 \\
\hline
Formal proof & 0 & 0 & 0 & 0 \\
\hline
\end{tabular}

\paragraph{Other elements for tool certification}

\section{Other comments}
Please to give free comments on the approach.

\begin{assessor1}
To me the main strengths of Papyrus are
\begin{itemize}
\item is that its underlying language, SysML, is rather standardized. 
\item its reliance on the Eclipse, and its strong organizational support, which leads to the integration of interesting features with Papyrus, and can foster the integration of additional ones. 
\end{itemize}

The main weaknesses of Papyrus are: 
\begin{itemize}
\item using Papyrus requires a high number of mouse click to perform relatively simple operations: many grop-down lists in the property edition panel in the editor. Modelers might be a bit frustrated by this edition overhead. 
\item it is not very clear what is to be considered as fully formal, and what is to consider as semi-formal.  
\end{itemize}
\end{assessor1}

