% Start here


\chapter{Conclusion}
\label{sec:concl}

This conclusion give a sum up of the evaluation results for each approach. The detailed results of each approach are given in the appendix.

\section{Main usage of the approach}
\label{main_usage}
This section discusses the main usage of the approach.

According to the figure \ref{fig:main_process}, for which phases do you recommend the approach (give a note from 0 to  3) :

\begin{tabular}{|l | c | c | c | c | c | c | c | c | c | c |}
\hline
&  \rotatebox{90}{GOPRR} & \rotatebox{90}{ERTMSFormalSpecs} &  \rotatebox{90}{SysML with Papyrus} &  \rotatebox{90}{SysML with Entreprise Architect} &  \rotatebox{90}{SCADE} &  \rotatebox{90}{EventB} &  \rotatebox{90}{Classical B} & \rotatebox{90}{Petri Nets} &  \rotatebox{90}{System C} &  \rotatebox{90}{GNATprove} \\
\hline 
System Analysis & 5 & & & & & & & & 2 & \\
\hline
Sub-system formal design  & 9 & & & & & & & & 4*  & \\
\hline
Software design  & 9 & & & & & & & & 9 & \\
\hline
Software code generation  & 9 & & & & & & & & 6 & \\
\hline
\end{tabular}

According to the figure \ref{fig:main_process}, for which type of activities do you recommend the approach (give a note from 0 to  3) :

\begin{tabular}{|l | c | c | c | c | c | c | c | c | c | c |}
\hline
& \rotatebox{90}{GOPRR} & \rotatebox{90}{ERTMSFormalSpecs} &  \rotatebox{90}{SysML with Papyrus} &  \rotatebox{90}{SysML with Entreprise Architect} &  \rotatebox{90}{SCADE} &  \rotatebox{90}{EventB} &  \rotatebox{90}{Classical B} & \rotatebox{90}{Petri Nets} &  \rotatebox{90}{System C} &  \rotatebox{90}{GNATprove} \\
\hline 
Documentation & 3 & & & & & & & & 0 & \\
\hline
Modeling & 9 & & & & & & & & 8 & \\
\hline
Design  & 6 & & & & & & & & 9 & \\
\hline
Code generation  & 9 & & & & & & & & 3* & \\
\hline
Verification  & 0 & & & & & & & & 2* & \\
\hline
Validation  & 0 & & & & & & & & 7 & \\
\hline
Safety analyses  & 0 & & & & & & & & 2* & \\
\hline
\end{tabular}

\section{Language}
This section discusses the main element of the language.

Which are the main characteristics of the language :

\begin{tabular}{|l | c | c | c | c | c | c | c | c | c | c |}
\hline
& \rotatebox{90}{GOPRR} & \rotatebox{90}{ERTMSFormalSpecs} &  \rotatebox{90}{SysML with Papyrus} &  \rotatebox{90}{SysML with Entreprise Architect} &  \rotatebox{90}{SCADE} &  \rotatebox{90}{EventB} &  \rotatebox{90}{Classical B} & \rotatebox{90}{Petri Nets} &  \rotatebox{90}{System C} &  \rotatebox{90}{GNATprove} \\
\hline 
Informal language & 0 & & & & & & & & 0 & \\
\hline 
Semi-formal language & 0 & & & & & & & & 9 & \\
\hline
Formal language & 8 & & & & & & & & 2 & \\
\hline
Structured language  & 9 & & & & & & & & 9 & \\
\hline
Modular language  & 9 & & & & & & & & 9 & \\
\hline
Textual language  & 0 & & & & & & & & 9 & \\
\hline
Mathematical symbols or code  & 0 & & & & & & & & 6 & \\
\hline
Graphical language  & 9 & & & & & & & & 0* & \\
\hline
\end{tabular}

According WP2 requirements, give a note for the capabilities of the language (from 0 to 3) :

\begin{tabular}{|l | c | c | c | c | c | c | c | c | c | c |}
\hline
& \rotatebox{90}{GOPRR} & \rotatebox{90}{ERTMSFormalSpecs} &  \rotatebox{90}{SysML with Papyrus} &  \rotatebox{90}{SysML with Entreprise Architect} &  \rotatebox{90}{SCADE} &  \rotatebox{90}{EventB} &  \rotatebox{90}{Classical B} & \rotatebox{90}{Petri Nets} &  \rotatebox{90}{System C} &  \rotatebox{90}{GNATprove} \\
\hline
Declarative formalization of properties (D.2.6-X-28)  & 7 & & & & & & & & 3 & \\
\hline
Simple formalization of properties (D.2.6-X-28.1)  & 6 & & & & & & & & 3 & \\
\hline
Scalability : capability to design large model  & 8 & & & & & & & & 9 & \\
\hline
Easily translatable to other languages (D.2.6-X-30)  & 9 & & & & & & & & 6 & \\
\hline
Executable directly (D.2.6-X-33)  & 0 & & & & & & & & 6* & \\
\hline
Executable after translation to a code (D.2.6-X-33)  & 9 & & & & & & & & 9 & \\
(precise if the translation is automatic)  & 8 & & & & & & & & 2* & \\
\hline
Simulation, animation (D.2.6-X-33)  & 0 & & & & & & & & 3 & \\
\hline
Easily understandable (D.2.6-X-27)  & 6 & & & & & & & & 7 & \\
\hline
Expertise level needed (0 High level, 3 few level)  & 6 & & & & & & & & 4 & \\
\hline
Standardization (D.2.6-X-29)  & * & & & & & & & & 9 & \\
\hline
Documented (D.2.6-X-29)  & 7 & & & & & & & & 9 & \\
\hline
Extensible language (D.2.6-01-28)  & 8 & & & & & & & & 9 & \\
\hline
\end{tabular}


\section{System Analysis}
This section discusses the usage of the approach for system analysis.
It can be skipped depending the results of \ref{main_usage}.

Acoording WP2 requirements, how the approach can be involved for the sub-system requirement specification ?

\begin{tabular}{|l | c | c | c | c | c | c | c | c | c | c |}
\hline
& \rotatebox{90}{GOPRR} & \rotatebox{90}{ERTMSFormalSpecs} &  \rotatebox{90}{SysML with Papyrus} &  \rotatebox{90}{SysML with Entreprise Architect} &  \rotatebox{90}{SCADE} &  \rotatebox{90}{EventB} &  \rotatebox{90}{Classical B} & \rotatebox{90}{Petri Nets} &  \rotatebox{90}{System C} &  \rotatebox{90}{GNATprove} \\
\hline
Independent System functions definition (D.2.6-X-10.2.1) & 6 & & & & & & & & - & \\
\hline 
System architecture design (D.2.6-X-10.2) & 9 & & & & & & & & - & \\
\hline
System data flow identification (D.2.6-X-10.2.3) & 9 & & & & & & & & - & \\
\hline
Sub-system focus (D.2.6-X-10.2.4) & 9 & & & & & & & & - & \\
\hline
System interfaces definition (D.2.6-X-10.2.5) & 9 & & & & & & & & - & \\
\hline
System requirement allocation (D.2.6-X-10.3) & 8 & & & & & & & & - & \\
\hline
Traceability with SRS (D.2.6-X-10.5) & 6 & & & & & & & & - & \\
\hline
Traceability with Safety activities (D.2.6-X-11) & 6 & & & & & & & & - & \\
\hline
\end{tabular}



\section{Sub-System formal design}
This section discusses the usage of the approach for sub-system formal design.
It can be skipped depending the results of \ref{main_usage}.

Two kinds of model can be planned during this phase: semi-formal models to  cover the SSRS (D.2.6-X-12.1) and strictly formal  models to  focuss on some functional and safety aspects (D.2.6-X-14).  Obviously some strictly  formal means can be used to define the semi-formal  model.

\subsection{Semi-formal model}

Concerning semi-formal model, how the WP2 requirements are covered ?

\begin{tabular}{|l | c | c | c | c | c | c | c | c | c | c |}
\hline
& \rotatebox{90}{GOPRR} & \rotatebox{90}{ERTMSFormalSpecs} &  \rotatebox{90}{SysML with Papyrus} &  \rotatebox{90}{SysML with Entreprise Architect} &  \rotatebox{90}{SCADE} &  \rotatebox{90}{EventB} &  \rotatebox{90}{Classical B} & \rotatebox{90}{Petri Nets} &  \rotatebox{90}{System C} &  \rotatebox{90}{GNATprove} \\
\hline 
Consistency to SSRS (D.2.6-X-12.2) & - & & & & & & & & 9 & \\
\hline
Coverage of SSRS (D.2.6-X-12.2.1) & - & & & & & & & & 2* & \\
\hline
Coverage of SSHA (D.2.6-X-12.2.2) & - & & & & & & & & 1* & \\
\hline
Management of requirement justification (D.2.6-X-12.2.3) & - & & & & & & & & 6 & \\
\hline
Traceability to  SSRS (D.2.6-X-12.2.5) & - & & & & & & & & 2* & \\
\hline
Traceability of exported requirements (D.2.6-X-12.2.6) & - & & & & & & & & 2* & \\
\hline
Simulation or animation (D.2.6-X-13 partial) & - & & & & & & & & 9 & \\
\hline
Execution (D.2.6-X-13 partial) & - & & & & & & & & 9 & \\
\hline
Extensible to strictly formal model (D.2.6-X-14.3) & - & & & & & & & & 4 & \\
\hline
Easy to  refine towards strictly formal model (D.2.6-X-14.4) & - & & & & & & & & 6 & \\
\hline
Extensible and modular design (D.2.6-X-15) & - & & & & & & & & 9 & \\
\hline
Extensible to software architecture and design (D.2.6-X-15) & - & & & & & & & & 9 & \\
\hline
\end{tabular}

Concerning safety properties management, how the WP2 requirements are covered ?

\begin{tabular}{|l | c | c | c | c | c | c | c | c | c | c |}
\hline
& \rotatebox{90}{GOPRR} & \rotatebox{90}{ERTMSFormalSpecs} &  \rotatebox{90}{SysML with Papyrus} &  \rotatebox{90}{SysML with Entreprise Architect} &  \rotatebox{90}{SCADE} &  \rotatebox{90}{EventB} &  \rotatebox{90}{Classical B} & \rotatebox{90}{Petri Nets} &  \rotatebox{90}{System C} &  \rotatebox{90}{GNATprove} \\
\hline 
Safety function isolation (D.2.6-X-17) & - & & & & & & & & * & \\
\hline 
Safety properties formalisation (D.2.6-X-22) & - & & & & & & & & 3* & \\
\hline
Logical expression (D.2.6-X-28.2.2) & -& & & & & & & & 9 & \\
\hline
Timing constraints (D.2.6-X-28.2.3) & - & & & & & & & & 9 & \\
\hline
Safety properties validation (D.2.6-X-23.2) & - & & & & & & & & 9 & \\
\hline
Logical properties assertion (D.2.6-X-34) & - & & & & & & & & 9 & \\
\hline
Check  of assertions (D.2.6-X-34.1) & - & & & & & & & & 9 & \\
\hline
\end{tabular}

Does the language allow to  formalize (D.2.6-X-31):

\begin{tabular}{|l | c | c | c | c | c | c | c | c | c | c |}
\hline
& \rotatebox{90}{GOPRR} & \rotatebox{90}{ERTMSFormalSpecs} &  \rotatebox{90}{SysML with Papyrus} &  \rotatebox{90}{SysML with Entreprise Architect} &  \rotatebox{90}{SCADE} &  \rotatebox{90}{EventB} &  \rotatebox{90}{Classical B} & \rotatebox{90}{Petri Nets} &  \rotatebox{90}{System C} &  \rotatebox{90}{GNATprove} \\
\hline 
State machines & - & & & & & & & & 8 & \\
\hline
Time-outs & - & & & & & & & & 9 & \\
\hline
Truth tables & - & & & & & & & & 9 & \\
\hline
Arithmetic & - & & & & & & & & 9 & \\
\hline
Braking curves & - & & & & & & & & 9 & \\
\hline
Logical statements & - & & & & & & & & 9 & \\
\hline
Message and fields & - & & & & & & & & 9 & \\
\hline
\end{tabular}


\subsection{Strictly formal model}

Concerning strictly formal model, how the WP2 requirements are covered ?

\begin{tabular}{|l | c | c | c | c | c | c | c | c | c | c |}
\hline
& \rotatebox{90}{GOPRR} & \rotatebox{90}{ERTMSFormalSpecs} &  \rotatebox{90}{SysML with Papyrus} &  \rotatebox{90}{SysML with Entreprise Architect} &  \rotatebox{90}{SCADE} &  \rotatebox{90}{EventB} &  \rotatebox{90}{Classical B} & \rotatebox{90}{Petri Nets} &  \rotatebox{90}{System C} &  \rotatebox{90}{GNATprove} \\
\hline 
Consistency to SFM (D.2.6-X-14.2) & * & & & & & & & & - & \\
\hline
Coverage of SSRS (D.2.6-X-14.2) & 3 & & & & & & & & - & \\
\hline
Traceability to  SSRS (D.2.6-X-14.3) & 7 & & & & & & & & - & \\
\hline
Extensible to software design (D.2.6-X-16) & 3 & & & & & & & & - & \\
\hline
Safety function isolation (D.2.6-X-17) & 3 & & & & & & & & - & \\
\hline 
Safety properties formalisation (D.2.6-X-22) & 5 & & & & & & & & - & \\
\hline
Logical expression (D.2.6-X-28.2.2) & 9 & & & & & & & & - & \\
\hline
Timing constraints (D.2.6-X-28.2.3) & 0 & & & & & & & & - & \\
\hline
Safety properties validation (D.2.6-X-23.3) & 7 & & & & & & & & - & \\
\hline
Logical properties assertion (D.2.6-X-34) & 9 & & & & & & & & - & \\
\hline
Proof of assertions (D.2.6-X-34.2) & 8 & & & & & & & & - & \\
\hline
\end{tabular}

Does the language allow to  formalize (D.2.6-X-32):

\begin{tabular}{|l | c | c | c | c | c | c | c | c | c | c |}
\hline
& \rotatebox{90}{GOPRR} & \rotatebox{90}{ERTMSFormalSpecs} &  \rotatebox{90}{SysML with Papyrus} &  \rotatebox{90}{SysML with Entreprise Architect} &  \rotatebox{90}{SCADE} &  \rotatebox{90}{EventB} &  \rotatebox{90}{Classical B} & \rotatebox{90}{Petri Nets} &  \rotatebox{90}{System C} &  \rotatebox{90}{GNATprove} \\
\hline 
State machines & 9 & & & & & & & & - & \\
\hline
Time-outs & 0 & & & & & & & & - & \\
\hline
Truth tables & 0 & & & & & & & & - & \\
\hline
Arithmetic & 9 & & & & & & & & - & \\
\hline
Braking curves & 9 & & & & & & & & - & \\
\hline
Logical statements & 9 & & & & & & & & - & \\
\hline
Message and fields & 9 & & & & & & & & - & \\
\hline
\end{tabular}


\section{Software design}
This section discusses the usage of the approach for software design.
It can be skipped depending the results of \ref{main_usage}.

\subsection{Functional design}

How the approach allows to  produce a functional software model of the on-board unit ?

\begin{tabular}{|l | c | c | c | c | c | c | c | c | c | c |}
\hline
& \rotatebox{90}{GOPRR} & \rotatebox{90}{ERTMSFormalSpecs} &  \rotatebox{90}{SysML with Papyrus} &  \rotatebox{90}{SysML with Entreprise Architect} &  \rotatebox{90}{SCADE} &  \rotatebox{90}{EventB} &  \rotatebox{90}{Classical B} & \rotatebox{90}{Petri Nets} &  \rotatebox{90}{System C} &  \rotatebox{90}{GNATprove} \\
\hline
Derivation from system semi-formal model & * & & & & & & & & 4* & \\
\hline 
Software architecture description & 9 & & & & & & & &  9 & \\
\hline
Software constraints & 9 & & & & & & & & 9 & \\
\hline
Traceability & 9 & & & & & & & & 6 & \\
\hline
Executable & 8 & & & & & & & & 9 & \\
\hline
\end{tabular}

\subsection{SSIL4 design}

How the approach allows to  produce in safety a software model ?

\begin{tabular}{|l | c | c | c | c | c | c | c | c | c | c |}
\hline
& \rotatebox{90}{GOPRR} & \rotatebox{90}{ERTMSFormalSpecs} &  \rotatebox{90}{SysML with Papyrus} &  \rotatebox{90}{SysML with Entreprise Architect} &  \rotatebox{90}{SCADE} &  \rotatebox{90}{EventB} &  \rotatebox{90}{Classical B} & \rotatebox{90}{Petri Nets} &  \rotatebox{90}{System C} &  \rotatebox{90}{GNATprove} \\
\hline
Derivation from system semi-formal or strictly formal model & * & & & & & & & & 2* & \\
\hline 
Software architecture description & 9 & & & & & & & & 9 & \\
\hline
Software constraints & 9 & & & & & & & & 9 & \\
\hline
Traceability & 9 & & & & & & & & 6 & \\
\hline
Executable & 8 & & & & & & & & 9 & \\
\hline
Conformance to EN50128 § 7.2 & 0 & & & & & & & & 0* & \\
\hline
Conformance to EN50128 § 7.3 & 0 & & & & & & & & 3* & \\
\hline
Conformance to EN50128 § 7.4 & 0 & & & & & & & & 3* & \\
\hline
\end{tabular}

Which criteria for software architecture are covered by the methodology
(see EN50128 table A.3) :

\begin{tabular}{|l | c | c | c | c | c | c | c | c | c | c |}
\hline
& \rotatebox{90}{GOPRR} & \rotatebox{90}{ERTMSFormalSpecs} &  \rotatebox{90}{SysML with Papyrus} &  \rotatebox{90}{SysML with Entreprise Architect} &  \rotatebox{90}{SCADE} &  \rotatebox{90}{EventB} &  \rotatebox{90}{Classical B} & \rotatebox{90}{Petri Nets} &  \rotatebox{90}{System C} &  \rotatebox{90}{GNATprove} \\
\hline
Defensive programming & 7 & & & & & & & & * & \\
\hline 
Fault detection \& diagnostic & 7 & & & & & & & & 6 & \\
\hline
Error detecting code & 7 & & & & & & & & 9 & \\
\hline
Failure assertion programming & 9 & & & & & & & & 9 & \\
\hline
Diverse programming & 3 & & & & & & & & * & \\
\hline
Memorising executed cases & 0 & & & & & & & & 9 & \\
\hline
Software error effect analysis & 0 & & & & & & & & 2* & \\
\hline
Fully defined interface & 9 & & & & & & & & 9 & \\
\hline
Modelling & 9 & & & & & & & & 5* & \\
\hline
Structured methodology & 9 & & & & & & & & 3* & \\
\hline
\end{tabular}

\section{Software code generation}
This section discusses the usage of the approach for software code generation.
It can be skipped depending the results of \ref{main_usage}.

Which criteria for software design and implementation are covered by the methodology
(see EN50128 table A.4) :

\begin{tabular}{|l | c | c | c | c | c | c | c | c | c | c |}
\hline
& \rotatebox{90}{GOPRR} & \rotatebox{90}{ERTMSFormalSpecs} &  \rotatebox{90}{SysML with Papyrus} &  \rotatebox{90}{SysML with Entreprise Architect} &  \rotatebox{90}{SCADE} &  \rotatebox{90}{EventB} &  \rotatebox{90}{Classical B} & \rotatebox{90}{Petri Nets} &  \rotatebox{90}{System C} &  \rotatebox{90}{GNATprove} \\
\hline
Formal methods & 6* & & & & & & & & 0 & \\
\hline 
Modeling & 6* & & & & & & & & 9 & \\
\hline
Modular approach (mandatory) & 9 & & & & & & & & 9 & \\
\hline
Components & 9 & & & & & & & & 9 & \\
\hline
Design and coding standards (mandatory) & 6* & & & & & & & & 6* & \\
\hline
Strongly typed programming language & 9 & & & & & & & & 6 & \\
\hline

\end{tabular}



\section{Main usage of the tool}
\label{main_usage}

This section discusses the main usage of the tool.

Which task are covered by the tool ?


\begin{tabular}{|l | c | c | c | c | c | c | c | c | c | c |}
\hline
& \rotatebox{90}{GOPRR} & \rotatebox{90}{ERTMSFormalSpecs} &  \rotatebox{90}{SysML with Papyrus} &  \rotatebox{90}{SysML with Entreprise Architect} &  \rotatebox{90}{SCADE} &  \rotatebox{90}{EventB} &  \rotatebox{90}{Classical B} & \rotatebox{90}{Petri Nets} &  \rotatebox{90}{System C} &  \rotatebox{90}{GNATprove} \\
\hline 
Modelling support & 9 & & & & & & & & 9 & \\
\hline
Automatic translation   & 3 & & & & & & & & 9 & \\
\hline
Code Generation   & 9 & & & & & & & & 2* & \\
\hline
Model verification  & 6 & & & & & & & & 4* & \\
\hline
Test generation  & * & & & & & & & & 2 & \\
\hline
Simulation, execution, debugging  & 6 & & & & & & & & 9 & \\
\hline
Formal proof  & 0 & & & & & & & & 5 & \\
\hline
\end{tabular}


\section{Use of the tool}

\begin{tabular}{|l | c | c | c | c | c | c | c | c | c | c |}
\hline
& \rotatebox{90}{GOPRR} & \rotatebox{90}{ERTMSFormalSpecs} &  \rotatebox{90}{SysML with Papyrus} &  \rotatebox{90}{SysML with Entreprise Architect} &  \rotatebox{90}{SCADE} &  \rotatebox{90}{EventB} &  \rotatebox{90}{Classical B} & \rotatebox{90}{Petri Nets} &  \rotatebox{90}{System C} &  \rotatebox{90}{GNATprove} \\
\hline 
Open Source (D2.6-X-36) & 0 & & & & & & & & 9 & \\
\hline 
Portability to operating systems (D2.6-X-37) & 9 & & & & & & & & 9 & \\
\hline
Cooperation of tools (D2.6-X-38) & 3 & & & & & & & & 4* & \\
\hline
Robustness (D2.6-X-41)  & 6* & & & & & & & & 9 & \\
\hline
Modularity (D2.6-X-41.1)  & 0 & & & & & & & & 9 & \\
\hline
Documentation management (D.2.6-X-41.2)  & 0 & & & & & & & & 4* & \\
\hline
Distributed software development (D.2.6-X-41.3)   & 9 & & & & & & & & 6* & \\
\hline
Simultaneous multi-users (D.2.6-X-41.4)   & 9 & & & & & & & & 9 & \\
\hline
Issue tracking (D.2.6-X-41.5)  & 0 & & & & & & & & 3* & \\
\hline
Differences between models (D.2.6-X-41.6)  & 0 & & & & & & & & 8 & \\
\hline
Version management (D.2.6-X-41.7)  & 3 & & & & & & & & 7 & \\
\hline
Concurrent version development (D.2.6-X-41.8)  & 0 & & & & & & & & 7 & \\
\hline
Model-based version control (D.2.6-X-41.9)  & 0 & & & & & & & & 0* & \\
\hline
Role traceability (D.2.6-X-41.10)  & 0 & & & & & & & & 0* & \\
\hline
Safety version traceability (D.2.6-X-41.11)  & 0 & & & & & & & & 0* & \\
\hline
Model traceability (D.2.6-01-035) & 3 & & & & & & & & 0* & \\
\hline
Tool chain integration  & 6* & & & & & & & & 8 & \\
\hline
Scalability  & 6* & & & & & & & & 9 & \\
\hline
\end{tabular}

\section{Certifiability}

This section discusses how the tool can be classified according EN50128 requirements (D.2.6-X-50).


\begin{tabular}{|l | c | c | c | c | c | c | c | c | c | c |}
\hline
& \rotatebox{90}{GOPRR} & \rotatebox{90}{ERTMSFormalSpecs} &  \rotatebox{90}{SysML with Papyrus} &  \rotatebox{90}{SysML with Entreprise Architect} &  \rotatebox{90}{SCADE} &  \rotatebox{90}{EventB} &  \rotatebox{90}{Classical B} & \rotatebox{90}{Petri Nets} &  \rotatebox{90}{System C} &  \rotatebox{90}{GNATprove} \\
\hline 
Tool manual (D.2.6-01-42.02) & 9 & & & & & & & & 6* & \\
\hline
Proof of correctness (D.2.6-01-42.03)    & 0 & & & & & & & & 0* & \\
\hline
Existing industrial  usage  & 9 & & & & & & & & 9 & \\
\hline
Model verification  & 3 & & & & & & & & 0* & \\
\hline
Test generation  & 0 & & & & & & & & 0* & \\
\hline
Simulation, execution, debugging  & 0 & & & & & & & & 9 & \\
\hline
Formal proof  & 0 & & & & & & & & 0* & \\
\hline
\end{tabular}
