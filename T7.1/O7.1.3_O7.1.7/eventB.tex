\chapter{EventB and Rodin}
\label{chap:eventB}

\begin{description}
\item[\textcolor{green}{Author}] Author of the approaches description  Matthias Gudemann - Marielle Petit-Doche (Systerel)
\item[\textcolor{blue}{Assessor 1}] First assessor of the approaches David Mentré (MERCE)
\item[\textcolor{magenta}{Assessor 2}] Second assessor of the approaches Laurent Ferrier (ERTMS Solutions)
\end{description}

In the sequel, main text is under the responsibilities of the author.

\begin{author_comment}
Author can add comments using this format at any place.
\end{author_comment}

\begin{assessor1}
First assessor can add comments using this format at any place.
\end{assessor1}

\begin{assessor2}
Second assessor can add comments using this format at any place.
\end{assessor2}

When a note is required, please follow this list :
\begin{description}
\item[0] not recommended, not adapted, rejected
\item[1] weakly recommended, adapted after major improvements, weakly rejected
\item[2] recommended, adapted (with light improvements if necessary)  weakly accepted
\item[3] highly recommended, well adapted,strongly accepted
\item[*] difficult to evaluate with a note (please add a comment under the table)
\end{description}

All the notes can be commented under each table.

\section{Presentation}

This section gives a quick presentation of the approach and the tool.

\begin{description}
\item[Name] Event-B method and Rodin tool
\item[Web site] \url{http://www.event-b.org/}
\item[Licence] open source (EPL licence)  available on source forge
\end{description}

\paragraph{Abstract} Short abstract on the approach and tool (10 lines max)

The formal language Event-B is based on a set-theoretic approach. It is a variant of the B
language, with a focus on system level modeling and has also be defined by  Jean-Raymond Abrial. An Event-B model is separated into a static
and a dynamic part.

Rodin is an industrial strength formal modeling tool. It allows the application of the Event-B approach to formal systems modeling. It provides proof obligation creation for invariants, refinement relations and data types. It comprises an Eclipse based modeling framework and supports numerous plugins, e.g., graphical modeling (iUML), automated proof support (theorem provers, SMT solvers) and traceability of requirements (ProR). It was developed by various academic and industrial partners in the European Union Projects RODIN (2003-2007), DEPLOY (2008-2012) and currently ADVANCE (2011-2014).

\paragraph{Publications} Short list of publications on the approach (5 max)

[Abrial2011]Modeling in Event-B: System and Software Engineering

Rodin Handbook (2012): \url{http://handbook.event-b.org}

[DEPLOY\_book2013] Industrial Deployment of System Engineering Methods \url{http://rodintools.org/flyer.pdf}


\section{Main usage of the approach}
\label{main_usage}
This section discusses the main usage of the approach.

According to the figure \ref{fig:main_process}, for which phases do you recommend the approach (give a note from 0 to  3) :

\begin{tabular}{|l | c | c | c | c|}
\hline
& \textcolor{green}{Author} & \textcolor{blue}{Assessor 1} & \textcolor{magenta}{Assessor 2} & Total \\
\hline 
System Analysis & 3 & 3 & &  \\
\hline
Sub-system formal design & 3 & 3 & & \\
\hline
Software design & 2 & 2 & & \\
\hline
Software code generation & 1 & 1 & & \\
\hline
\end{tabular}

\begin{author_comment}
This approach is designed for analyses at the early stage of the development of a system. Its capabilities of abstraction allow easy system level reasoning, without taking into account implementation details. Classical B, based on the same language is better adapted for software design and code generation \ref{chap:classicalB}.

It is possible to generate C or Ada code from an Event-B model \url{http://eprints.soton.ac.uk/336226/1/ABZ2012_short_v20120202.pdf} and \url{http://deploy-eprints.ecs.soton.ac.uk/375/1/AdaEurope2012.pdf}. Two  approaches are on development: \url{http ://wiki.event-b.org/index.php/Code_Generation_Activity} and \url{http://eb2all.loria.fr/}. However, the Event-B specification must be sufficiently detailed to generate code, and this code generator do not take care of software constraints on critical systems or code optimisation.
\end{author_comment}


According to the figure \ref{fig:main_process}, for which type of activities do you recommend the approach (give a note from 0 to  3) :

\begin{tabular}{|l | c | c | c | c|}
\hline
& \textcolor{green}{Author} & \textcolor{blue}{Assessor 1} & \textcolor{magenta}{Assessor 2} & Total \\
\hline 
Documentation & 2 & 2 & &  \\
\hline
Modeling & 3 & 3 & &  \\
\hline
Design & 3 & 3 & & \\
\hline
Code generation & 1 & 1 & & \\
\hline
Verification & 3 & 3 & & \\
\hline
Validation & 3 & 3 & & \\
\hline
Safety analyses & 2 & 2 & & \\
\hline
\end{tabular}

\begin{author_comment}
This approach is not sufficient to cover all the safety analyses activities. But it is useful to give confidence on the analyses on a mathematical model. See \url{http://www.erts2012.org/Site/0P2RUC89/4D-2.pdf}.
\end{author_comment}


\paragraph{Known usages} Have you some examples of usage of this approach to  compare with the OpenETCS objectives ?


\begin{author_comment}
This approach has already been used in railway on two  main critical systems: modelling the DIR41 (instead the generic specification of all interlocking systems used in the metro in Paris)  and modelling the track-side controller from the metro of Lyon. 

There are others application in the automotive and aeronautic domains. See for more details: \url{http://wiki.event-b.org/index.php/Industrial_Projects}
\end{author_comment}


\section{Language}
This section discusses the main element of the language.

Which are the main characteristics of the language :

\begin{tabular}{|l | c | c | c | c|}
\hline
& \textcolor{green}{Author} & \textcolor{blue}{Assessor 1} & \textcolor{magenta}{Assessor 2} & Total \\
\hline 
Informal language & 0 & 0 & &  \\
\hline 
Semi-formal language & 0 & 0 & &  \\
\hline
Formal language & 3 & 3 & &  \\
\hline
Structured language & 3 & 3 & & \\
\hline
Modular language & 3 & 2 & & \\
\hline
Textual language & 0 & 3 & & \\
\hline
Mathematical symbols or code & 3 & 3 & & \\
\hline
Graphical language & 2 & 2 & & \\
\hline
\end{tabular}

\begin{author_comment}
Event-B  is a formal language with a mathematical notation. But the Rodin tool can be completed with plug-in like iUML \url{http://wiki.event-b.org/index.php/IUML-B} allows to model directly state-machine or block diagram  graphically and to  give the event-B mathematical translation.
\end{author_comment}

According WP2 requirements, give a note for the capabilities of the language (from 0 to 3) :

\begin{tabular}{|l | c | c | c | c|}
\hline
& \textcolor{green}{Author} & \textcolor{blue}{Assessor 1} & \textcolor{magenta}{Assessor 2} & Total \\
\hline
Declarative formalization of properties (D.2.6-X-28) & 3 & 3 & & \\
\hline
Simple formalization of properties (D.2.6-X-28.1) & 3 & 2 & & \\
\hline
Scalability : capability to design large model & 2 & 2 & & \\
\hline
Easily translatable to other languages (D.2.6-X-30) & 3 & 3 & & \\
\hline
Executable directly (D.2.6-X-33) & 0 & 0 & & \\
\hline
Executable after translation to a code (D.2.6-X-33) & 2 & 2 & & \\
(precise if the translation is automatic) & & & & \\
\hline
Simulation, animation (D.2.6-X-33) & 3 & 3 & & \\
\hline
Easily understandable (D.2.6-X-27) & 3 & 2 & & \\
\hline
Expertise level needed (0 High level, 3 few level) & 1 & 1 & & \\
\hline
Standardization (D.2.6-X-29) & 1 & 1 & & \\
\hline
Documented (D.2.6-X-29) & 3 & 3 & & \\
\hline
Extensible language (D.2.6-01-28) & 3 & - & & \\
\hline
\end{tabular}


\paragraph{Documentation} Describe how the language is documented, the existing guidelines, coding rules, standardization...

\begin{author_comment}
A description of the language is available here \url{http://wiki.event-b.org/index.php/Event-B_Language}.

A complete handbook of the Rodin tool is available  here \url{http://handbook.event-b.org/}.

Currently there is no guidelines or coding rules.
\end{author_comment}

\paragraph{Language usage} Describe the possible restriction on the language

\begin{author_comment}
No restriction known.
\end{author_comment}

\section{System Analysis}
This section discusses the usage of the approach for system analysis.
It can be skipped depending the results of \ref{main_usage}.

Acoording WP2 requirements, how the approach can be involved for the sub-system requirement specification ?

\begin{tabular}{|l | c | c | c | c|}
\hline
& \textcolor{green}{Author} & \textcolor{blue}{Assessor 1} & \textcolor{magenta}{Assessor 2} & Total \\
\hline
Independent System functions definition (D.2.6-X-10.2.1)  & 3 & 3 & &  \\
\hline 
System architecture design (D.2.6-X-10.2) & 3 & 3 & &  \\
\hline
System data flow identification (D.2.6-X-10.2.3)  & 3 & 2 & &  \\
\hline
Sub-system focus (D.2.6-X-10.2.4)  & 3 & 3 & &  \\
\hline
System interfaces definition (D.2.6-X-10.2.5)  & 3 & 3 & &  \\
\hline
System requirement allocation (D.2.6-X-10.3)  & 3 & 3 & &  \\
\hline
Traceability with SRS (D.2.6-X-10.5)  & 3 & 3 & &  \\
\hline
Traceability with Safety activities (D.2.6-X-11)  & 3 & 3 & &  \\
\hline
\end{tabular}

\begin{author_comment}
Requirement traceability can be manage with the ProR plug-in \url{http://wiki.event-b.org/index.php/ProR}.
\end{author_comment}

\section{Sub-System formal design}
This section discusses the usage of the approach for sub-system formal design.
It can be skipped depending the results of \ref{main_usage}.

Two kinds of model can be planned during this phase: semi-formal models to  cover the SSRS (D.2.6-X-12.1) and strictly formal  models to  focuss on some functional and safety aspects (D.2.6-X-14).  Obviously some strictly  formal means can be used to define the semi-formal  model.

\subsection{Semi-formal model}


\begin{author_comment}
As a formal language, Event-B  can cover some artifacts of semi-formal models.
\end{author_comment}

Concerning semi-formal model, how the WP2 requirements are covered ?

\begin{tabular}{|l | c | c | c | c|}
\hline
& \textcolor{green}{Author} & \textcolor{blue}{Assessor 1} & \textcolor{magenta}{Assessor 2} & Total \\
\hline 
Consistency to SSRS (D.2.6-X-12.2) & 2 & 2 & &  \\
\hline
Coverage of SSRS (D.2.6-X-12.2.1)  & 2 & 2 & &  \\
\hline
Coverage of SSHA (D.2.6-X-12.2.2)  & 2 & 2 & &  \\
\hline
Management of requirement justification (D.2.6-X-12.2.3)  & 3 & 3 & &  \\
\hline
Traceability to  SSRS (D.2.6-X-12.2.5)  & 3 & 3 & &  \\
\hline
Traceability of exported requirements (D.2.6-X-12.2.6)  & 3 & 3 & &  \\
\hline
Simulation or animation (D.2.6-X-13 partial)  & 3 & 3 & &  \\
\hline
Execution (D.2.6-X-13 partial)  & 1 & 1 & &  \\
\hline
Extensible to strictly formal model (D.2.6-X-14.3) & 3 & 3 & &  \\
\hline
Easy to  refine towards strictly formal model (D.2.6-X-14.4) & 3 & 3 & &  \\
\hline
Extensible and modular design (D.2.6-X-15)  & 3 & 2 & &  \\
\hline
Extensible to software architecture and design (D.2.6-X-30)   & 3 & 3 & &  \\
\hline
\end{tabular}


\begin{author_comment}
Model can be executed after translation.
\end{author_comment}

Concerning safety properties management, how the WP2 requirements are covered ?

\begin{tabular}{|l | c | c | c | c|}
\hline
& \textcolor{green}{Author} & \textcolor{blue}{Assessor 1} & \textcolor{magenta}{Assessor 2} & Total \\
\hline 
Safety function isolation (D.2.6-X-17)  & 3 & 3 & &  \\
\hline 
Safety properties formalisation (D.2.6-X-22)  & 3 & 3 & &  \\
\hline
Logical expression (D.2.6-X-28.2.2)  & 3 & 3 & &  \\
\hline
Timing constraints (D.2.6-X-28.2.3)  & 1 & 1 & &  \\
\hline
Safety properties validation (D.2.6-X-23.2)  & 2 & 3 & &  \\
\hline
Logical properties assertion (D.2.6-X-34)  & 3 & 3 & &  \\
\hline
Check  of assertions (D.2.6-X-34.1)  & 3 & 3 & &  \\
\hline
\end{tabular}


\begin{author_comment}
Concerning time-constraints, it is possible to  model  abstract properties as time-outs, but not performance constraints.
\end{author_comment}


Does the language allow to  formalize (D.2.6-X-31):

\begin{tabular}{|l | c | c | c | c|}
\hline
& \textcolor{green}{Author} & \textcolor{blue}{Assessor 1} & \textcolor{magenta}{Assessor 2} & Total \\
\hline 
State machines  & 3 & 3 & &  \\
\hline
Time-outs  & 2 & 2 & &  \\
\hline
Truth tables  & 3 & 3 & &  \\
\hline
Arithmetic  & 3 & 3 & &  \\
\hline
Braking curves  & 1 & 1 & &  \\
\hline
Logical statements & 3 & 3 & &  \\
\hline
Message and fields & 3 & 2 & &  \\
\hline
\end{tabular}

\paragraph{Additional comments on semi-formal  model} Do you think your semi-formal  model is sufficient to cover a safe design of the on-board unit until code generation ?
All comments on links to  other models, validation and verification activities are welcomed.


\begin{author_comment}
Event-B  approach is well adapted for system analysis and design. However other formal approach as Classical-B \ref{chap:classicalB} or Scade are more relevant to be used during software design to code generation of critical systems.
\end{author_comment}


\begin{assessor1}
I'm not sure Event-B is suitable for modelling ``message and fields''.
\end{assessor1}


\subsection{Strictly formal model}

Concerning strictly formal model, how the WP2 requirements are covered ?

\begin{tabular}{|l | c | c | c | c|}
\hline
& \textcolor{green}{Author} & \textcolor{blue}{Assessor 1} & \textcolor{magenta}{Assessor 2} & Total \\
\hline 
Consistency to SFM (D.2.6-X-14.2) & 3 & 3 & &  \\
\hline
Coverage of SSRS (D.2.6-X-14.2)  & 2 & 2 & &  \\
\hline
Traceability to  SSRS (D.2.6-X-14.3)  & 3 & 3 & &  \\
\hline
Extensible to software design (D.2.6-X-16)  & 3 & 2 & &  \\
\hline
Safety function isolation (D.2.6-X-17)  & 3 & 3 & &  \\
\hline 
Safety properties formalisation (D.2.6-X-22)  & 3 & 3 & &  \\
\hline
Logical expression (D.2.6-X-28.2.2)  & 3 & 3 & &  \\
\hline
Timing constraints (D.2.6-X-28.2.3)  & 2 & 1 & &  \\
\hline
Safety properties validation (D.2.6-X-23.3)  & 2 & 3 & &  \\
\hline
Logical properties assertion (D.2.6-X-34)  & 3 & 3 & &  \\
\hline
Proof of assertions (D.2.6-X-34.2)  & 3 & 3 & &  \\
\hline
\end{tabular}

Does the language allow to  formalize (D.2.6-X-32):

\begin{tabular}{|l | c | c | c | c|}
\hline
& \textcolor{green}{Author} & \textcolor{blue}{Assessor 1} & \textcolor{magenta}{Assessor 2} & Total \\
\hline 
State machines  & 3 & 3 & &  \\
\hline
Time-outs  & 2 & 1 & &  \\
\hline
Truth tables  & 3 & 3 & &  \\
\hline
Arithmetic  & 3 & 3 & &  \\
\hline
Braking curves  & 1 & 1 & &  \\
\hline
Logical statements & 3 & 3 & &  \\
\hline
Message and fields & 3 & 2 & &  \\
\hline
\end{tabular}

\paragraph{Additional comments on semi-formal  model} Do you think your strictly formal  model can be directly defined from the SSRS ?
All comments on links to  other models, validation and verification activities are welcomed.


\begin{author_comment}
Event-B  approach is well adapted for system analysis and design. However other formal approach as Classical-B \ref{chap:classicalB} or Scade are more relevant to be used during software design to code generation of critical systems.
\end{author_comment}


\section{Software design}
This section discusses the usage of the approach for software design.
It can be skipped depending the results of \ref{main_usage}.


\begin{author_comment}
This section is skipped :  for software design classical B is more adapted than event B \ref{chap:classicalB}.
\end{author_comment}


\subsection{Functional design}


How the approach allows to  produce a functional software model of the on-board unit ?

\begin{tabular}{|l | c | c | c | c|}
\hline
& \textcolor{green}{Author} & \textcolor{blue}{Assessor 1} & \textcolor{magenta}{Assessor 2} & Total \\
\hline
Derivation from system semi-formal model  & & & &  \\
\hline 
Software architecture description  & & & &  \\
\hline
Software constraints  & & & &  \\
\hline
Traceability  & & & &  \\
\hline
Executable  & & & &  \\
\hline
\end{tabular}

\subsection{SSIL4 design}

How the approach allows to  produce in safety a software model ?

\begin{tabular}{|l | c | c | c | c|}
\hline
& \textcolor{green}{Author} & \textcolor{blue}{Assessor 1} & \textcolor{magenta}{Assessor 2} & Total \\
\hline
Derivation from system semi-formal or strictly formal model  & & & &  \\
\hline 
Software architecture description  & & & &  \\
\hline
Software constraints  & & & &  \\
\hline
Traceability  & & & &  \\
\hline
Executable  & & & &  \\
\hline
Conformance to EN50128 § 7.2  & & & &  \\
\hline
Conformance to EN50128 § 7.3  & & & &  \\
\hline
Conformance to EN50128 § 7.4  & & & &  \\
\hline
\end{tabular}

Which criteria for software architecture are covered by the methodology
(see EN50128 table A.3) :

\begin{tabular}{|l | c | c | c | c|}
\hline
& \textcolor{green}{Author} & \textcolor{blue}{Assessor 1} & \textcolor{magenta}{Assessor 2} & Total \\
\hline
Defensive programming  & & & &  \\
\hline 
Fault detection \& diagnostic  & & & &  \\
\hline
Error detecting code  & & & &  \\
\hline
Failure assertion programming & & & &  \\
\hline
Diverse programming & & & &  \\
\hline
Memorising executed cases & & & &  \\
\hline
Software error effect analysis & & & &  \\
\hline
Fully defined interface & & & &  \\
\hline
Modelling  & & & &  \\
\hline
Structured methodology & & & &  \\
\hline
\end{tabular}

\section{Software code generation}
This section discusses the usage of the approach for software code generation.
It can be skipped depending the results of \ref{main_usage}.


\begin{author_comment}
This section is skipped :  for software design classical B is more adapted than event B \ref{chap:classicalB}.
\end{author_comment}


Which criteria for software design and implementation are covered by the methodology
(see EN50128 table A.4) :

\begin{tabular}{|l | c | c | c | c|}
\hline
& \textcolor{green}{Author} & \textcolor{blue}{Assessor 1} & \textcolor{magenta}{Assessor 2} & Total \\
\hline
Formal methods  & & & &  \\
\hline 
Modeling  & & & &  \\
\hline
Modular approach (mandatory) & & & &  \\
\hline
Components & & & &  \\
\hline
Design and coding standards (mandatory) & & & &  \\
\hline
Strongly typed programming language & & & &  \\
\hline

\end{tabular}



\section{Main usage of the tool}
\label{main_usage}

This section discusses the main usage of the tool.

Which task are covered by the tool ?


\begin{tabular}{|l | c | c | c | c|}
\hline
& \textcolor{green}{Author} & \textcolor{blue}{Assessor 1} & \textcolor{magenta}{Assessor 2} & Total \\
\hline 
Modelling support & 3 & 3 & &  \\
\hline
Automatic translation  & 3 & 2 & & \\
\hline
Code Generation  & 1 & 1 & & \\
\hline
Model verification & 3 & 3 & & \\
\hline
Test generation & 1 & 1 & & \\
\hline
Simulation, execution, debugging & 3 & 2 & & \\
\hline
Formal proof & 3 & 3 & & \\
\hline
\end{tabular}

\paragraph{Modelling support}
Does the tool provide a  textual or a graphical editor ?


\begin{author_comment}
The tool provide a textual support, however some plug-ins allow to  model graphical diagrams (as UML or SysML diagram) and to give translation in event-B textual language \url{http://wiki.event-b.org/index.php/IUML-B}
\end{author_comment}



\paragraph{Automatic translation and code generation}
Which translation or code generation is supported by the tool ?

\begin{author_comment}
Automatic translation are possible from SysML  or UML diagram to event-B, and from event-B to C, C++  or Ada code, see \url{http://wiki.event-b.org/index.php/Code_Generation_Activity} and \url{http://eb2all.loria.fr/}.
\end{author_comment}

\paragraph{Model verification}
Which verification on models are provided by the tool?

\begin{author_comment}

Syntax, type verification,
refinement verification
formal proof of properties (validation)
\end{author_comment}

\paragraph{Test generation}
Does the tool allow to generate tests ? For  which purpose ?

\begin{author_comment}

The tool is not design to generate test, however we can imagine means to  generate test from event-B machine (Rodin can be linked to the model checker ProB).
\end{author_comment}

\paragraph{Simulation, execution, debugging}
Does the tool allow to simulate or to debbug step by step a model or a code ?

\begin{author_comment}

Two plug-ins (ProB \url{http://wiki.event-b.org/index.php/ProB} and animateB \url{http://wiki.event-b.org/index.php/AnimB}) are currently existing to simulate and animate the Event-B models. It is also possible to animate state machines\url{http://wiki.event-b.org/index.php/UML-B_-_Statemachine_Animation}.
\end{author_comment}

\paragraph{Formal proof}
Does the tool allow formal proof ?  How ?

\begin{author_comment}

Formal proof is one of the most important artefacts of the tool. different elements take part of the formal proof:
\begin{itemize}
\item proof generator
\item interactive prover
\item automatic prover
\item plug-in to link to SAT solver
\item plug-in to model-checker as ProB
\item plug-in to AtelierB prover
\end{itemize}
\end{author_comment}


\section{Use of the tool}


According WP2 requirements, give a note for characteristics of the use of the tool (from 0 to 3) :

\begin{tabular}{|l | c | c | c | c|}
\hline
& \textcolor{green}{Author} & \textcolor{blue}{Assessor 1} & \textcolor{magenta}{Assessor 2} & Total \\
\hline 
Open Source (D2.6-X-36) & 3 & 2 & &  \\
\hline 
Portability to operating systems (D2.6-X-37) & 3 & 3 & &  \\
\hline
Cooperation of tools (D2.6-X-38) & 3 & 3 & &  \\
\hline
Robustness (D2.6-X-41) & 2 & 2 & & \\
\hline
Modularity (D2.6-X-41.1) & 3 & 3 & & \\
\hline
Documentation management (D.2.6-X-41.2) & 3 & 3 & & \\
\hline
Distributed software development (D.2.6-X-41.3)  & 2 & 2 & & \\
\hline
Simultaneous multi-users (D.2.6-X-41.4)   & 1 & 1 & & \\
\hline
Issue tracking (D.2.6-X-41.5) & 0 & 0 & & \\
\hline
Differences between models (D.2.6-X-41.6) & 2 & & & \\
\hline
Version management (D.2.6-X-41.7) & 3 & 1 & & \\
\hline
Concurrent version development (D.2.6-X-41.8) & 2 & 1 & & \\
\hline
Model-based version control (D.2.6-X-41.9) & 2 & 2 & & \\
\hline
Role traceability (D.2.6-X-41.10) & 1 & 1 & & \\
\hline
Safety version traceability (D.2.6-X-41.11) & 1 & 1 & & \\
\hline
Model traceability (D.2.6-01-035) & 3 & 2 & & \\
\hline
Tool chain integration & 3 & 3 & & \\
\hline
Scalability & 2 & 2 & & \\
\hline
\end{tabular}

\begin{author_comment}
The tool is based on Eclipse on EUPL licence. Thus is can be associated to  other tools on eclipse easily, for example: EMF model comparator, Git plug-in,...
\end{author_comment}


\section{Certifiability}

This section discusses how the tool can be classified according EN50128 requirements (D.2.6-X-50).


\begin{tabular}{|l | c | c | c | c|}
\hline
& \textcolor{green}{Author} & \textcolor{blue}{Assessor 1} & \textcolor{magenta}{Assessor 2} & Total \\
\hline 
Tool manual (D.2.6-01-42.02) & 3 & 3 & &  \\
\hline
Proof of correctness (D.2.6-01-42.03)   & 0 & 0 & & \\
\hline
Existing industrial  usage  & 2 & 2 & & \\
\hline
Model verification & 3 & 3 & & \\
\hline
Test generation & 0 & 0 & & \\
\hline
Simulation, execution, debugging & 3 & 2 & & \\
\hline
Formal proof &3  & 3 & & \\
\hline
\end{tabular}

\paragraph{Other elements for tool certification}

\begin{author_comment}
Event-B method is quoted and recommended for system analyses in the draft version of future standard EN50126.

\end{author_comment}

\section{Other comments}
Please to  give free comments on the approach.



